\chapter{Data Analysis}
I will describe the procedure of identifying the secondary beam of ${}^{17}$B and choose the event of fragment ${}^{15}$B and two neutrons.

\section{Analysis of the secondary beam}
The primary beam is generated by SRC, RIBF. By accelerating ${}^{48}$Ca in  MeV/u, bombarded on thick Be target (cm) secondary beam is produced. \\
Identification of ${}^{17}$B secondary beam  is performed by TOF-B$\rho$-$\Delta$E method.

\subsection{Beam Energy Calculation}
Brho from F5,

\subsection{Plastic Scintillator Gate}

\subsection{TDC Calibration}
\subsection{BDC Profile}
\subsection{Target Profile}
\subsection{Beam beta calculation}
Using LISE++, 

\section{Analysis of charged fragment}
\subsection{FDC를 이용한 위치 계산}
\subsection{Brho from Geant4 simulation}
\subsection{HODscope Z}
15B를 특정하기 위하여, Hodscope라는 Plastic Scintillator 검출기에서 Z를 특정한다.
상류에서 17B를 특정하여, 가장 많은 입자가 Z=5라고 가정, 위의 시뮬레이션으로 얻은 AOZ를 대입하여 15B를 특정한다.
\subsection{Fragment Particle Identification}

\clearpage

\section{Analysis of Neutrons}
2차 빔 ${}^{17}$B에서 방출된 중성자는 사무라이 시스템의 중성자 검출기 NEBULA를 통해 검출된다. 본 섹션에서는 중성자 검출기 NEBULA를 이용하여 1중성자 이벤트 혹은 2중성자 이벤트를 선택하는 과정에 대해 서술한다.
중성자의 운동량 모멘텀 $P(n)$은 타겟으로부터의 NEBULA에서의 검출 위치, 그리고 비행시간 TOF로 Reconstructed. 중성자의 은 다음과 같이 기술된다.


\section{Cross-talk Rejection}
(중성자의 검출 방법 서술) 두 중성자의 이벤트를 추출하는 과정에서 가장 중요한 것이 Cross-talk 제거이다. Cross-talk란, 하나의 중성자가 여러개의 신호를 만드는 것으로, 두 중성자 이벤트를 선택할 때에 있어 가장 많은 노이즈를 차지한다. 
본 실험에 있어서, 두 중성자는 중성자 검출기 NEBULA에 검출된 가장 빠른 두 이벤트를 선택하고, 

\subsection{gamma event rejection}

\subsection{same wall event}
두 중성자

\subsubsection{cluster proton cross-talk}


\subsection{different wall}

\section{Acceptance and Efficiency}
NEBULA의 Acceptance and Efficiency를 평가하기 위해 Geant4 Simulation을 진행했다. 