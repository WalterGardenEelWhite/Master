\appendix

\chapter{Constants}
\section{Mass}
For calculating the mass of the nucleus, the mass excess is used. The mass is defined with mass excess ($\Delta$) as
\begin{align}
    M(A,Z) = A m_u + \Delta(A,Z),
\end{align}
where $m_u$ is the atomic mass unit. In this thesis, the mass excess of each nucleus is used from the ~~
\begin{table}[h]
    \centering
    \begin{tabular}{c|c}
        \hline \hline
        Nucleus & Mass Excess(MeV) \\
        \hline
        ${}^{19}$B & 59.8 \\
        ${}^{17}$B & 43.72 \\
        ${}^{15}$B & 28.957 \\
        ${}^{14}$B & 23.664 \\
        ${}^{13}$B & 16.561 \\
        \hline \hline       
    \end{tabular}
\end{table}

\chapter{Equations}
\section{Bethe-Bloch Formula}
Energy loss calculation for reconstruction of invariant mass is done by Bethe-Bloch formula. 

\section{Equivalent Photon Method}
considering multi-polarity, 
\begin{align}
    n_{\pi l}(\omega)=Z^{2}_{1} \alpha \frac{l[(2l+1)]!!^{2}}{(2\pi)^{3}(l+1)}\sum_{m}\Big|G_{\pi l m}\big( \frac{c}{v}\big) \Big|^{2} g_{m}(\xi)
\end{align} 

equivalent photon number for E1 excited projectile.
\begin{align}
    n_{E1}(\omega)=n_{E1, m=-1}+n_{E1, m=0}+n_{E1, m=+1}\\
    =\frac{2}{\pi}Z^{2}_{1}\alpha\Big(\frac{c}{v}\Big)^{2}\Big[ \xi K_{0}(\xi)K_{1}(\xi)-\frac{v^{2}\xi^{2}}{2c^{2}}(K^{2}_{1}(\xi)-K^{2}_{0}(\xi)) \Big]
\end{align}

with 
\begin{align}
    \xi=E_{\gamma}R/\gamma \nu \hbar\\
    E_{\gamma}: Photon energy( E_{\gamma}=\omega\hbar )\\
    Z_{1} Atomic number of target
\end{align}

%R: impact parameter ( 1.3 )

\chapter{Resolution Evaluation}
\section{Intrinsic Resolution of Drift Chambers}

\section{Position and Angular resolution at the target}
Using intrinsic resolution of each drift chamber, the position and angular resolution at the target can be calculated. The angular resolution at the target is calculated as
\begin{align}
    \Delta \theta_x^{\text{tgt}} = \Delta \bigg( \frac{x_{\text{BDC2}} - x_{\text{BDC1}}}{L(\text{BDC2} - \text{BDC1})} \bigg) 
    = \frac{\sqrt{(\Delta x_{\text{BDC2}})^2 + (\Delta x_{\text{BDC1}})^2}}{1000 \text{mm}} \\ \notag
    \Delta \theta^{\text{tgt}}_y = \Delta \Big( \frac{y_{\text{BDC2}} - y_{\text{BDC1}}}{L(\text{BDC2} - \text{BDC1})} \Big) 
    = \frac{\sqrt{(\Delta y_{\text{BDC2}})^2 + (\Delta y_{\text{BDC1}})^2}}{1000 \text{mm}}
\end{align}
The position resolution at the target is
\begin{align}
    \Delta x_{\text{tgt}} = \sqrt{(\Delta x_{\text{BDC2}})^2 + (\Delta \theta_x^{\text{tgt}} \cdot L(\text{tgt} - \text{BDC2}))^2} \\ \notag
    \Delta y_{\text{tgt}} = \sqrt{(\Delta y_{\text{BDC2}})^2 + (\Delta \theta_y^{\text{tgt}} \cdot L(\text{tgt} - \text{BDC2}))^2}
\end{align}


\section{Two-body Relative Energy Resolution}


\chapter{}
\begin{table}[h]
    \centering
    \begin{tabular}[h]{c|c}
        \hline
        Flight length & Distance \\
        \hline
        F7-F13 & 7.5m \\
        dist-BDC1-BDC2 & 1.5m \\
        dist-BDC1-tgt & 1.5m \\
        target z & 0.5m \\
        dist-FDC1-Tgt & 1.5m \\
        dist-FDC2-HOD & 1.5m \\
        \hline
    \end{tabular}
    \caption[short]{Flight length}
\end{table}


\begin{figure}[h]
    \centering
    \setlength{\unitlength}{1mm}
    \begin{picture}(100,30)
      % GDR
      \put(70,20){\circle{20}} % GDR
      \put(72,20){\circle{20}} % GDR
      \put(70,20){\vector(1,0){3}}%
      \put(70,20){\vector(-1,0){2}}%
      \put(63,30){$p$} % p label
      \put(77,30){$n$} % n label
      \put(67,5){GDR} % GDR label
    
      % Soft E1
      \put(20,20){\circle{18}}
      \put(23,20){\circle*{10}}
        \put(15,20){\vector(1,0){2}}
        \put(15,20){\vector(-1,0){2}}
        \put(15,5){Soft $E1$}
        
    \end{picture}
    \caption{The schematic representation of the giant dipole resonance and soft dipole mode}
    \end{figure}


    \begin{align}
        \langle \text{cos} \theta_{nn} \rangle &= \langle \Psi({}^{11}\text{Li}) | \text{cos} \theta_{nn} | \Psi({}^{11}\text{Li}) \rangle \notag \\
        &= \alpha^2 \langle (1d_{5/2})^2 | \text{cos} \theta_{nn} | (1d_{5/2})^2 \rangle + \beta^2 \langle (2s_{1/2})^2 | \text{cos} \theta_{nn} | (2s_{1/2})^2 \rangle \notag \\
        &\hspace{4mm} + 2\alpha \beta \langle (1d_{5/2})^2 | \text{cos} \theta_{nn} | (2s_{1/2})^2 \rangle \notag \\
        &= 2\alpha \beta \langle (1d_{5/2})^2 | \text{cos} \theta_{nn} | (2s_{1/2})^2 \rangle
    \end{align}
    
    where $|(1d_{5/2}^2)$ represents $|\psi({}^{15}\text{B})\otimes|$
    \begin{align}
        \langle r^2_m \rangle = \frac{A_c}{A}\langle r^2_m \rangle_c + \frac{2A_c}{A^2}\langle r^2_{c-nn} \rangle + \frac{1}{2A}\langle r^2_{nn} \rangle
    \end{align}