\chapter{Experiment}
This chapter describes the experimental setup in this study. The experiment is performed at Rare Isotope Beam Factory (RIBF) at RIKEN Nishina Center \cite{RIKEN}. A primary ${}^{48}$Ca beam is produced by the RIKEN accelerator complex and delivered to the BigRIPS separator \cite{Kubo03}\cite{Kubo07}\cite{Kubo12}. The BigRIPS separator produced the ${}^{17}$B secondary beam which bombarded on the lead and carbon targets in front of the SAMURAI (Superconducting Analyzer for MUlti-particles from Radioisotope beams) spectrometer\cite{SAMURAI}. After the reaction at the target, the fragments ${}^{15}$B and two neutrons are detected by detectors at SAMURAI area. 



The primary beam was accelerated by the SRC, RIBF. By accelerating ${}^{48}$Ca to 345 MeV/u and bombarding it on a thick Be target (30mm), a secondary beam was produced. The characteristics of the primary beam is in Table \ref{tab:Primary_Beam}.

    \begin{table}[h]
        \centering 
            \begin{tabular}[]{c|c|c|c}
                \hline
                Primary Beam & Beam Energy & Beam Intensity & Primary Target  \\
                \hline 
                ${}^{48}$Ca & 345 MeV/u & 210 pnA & Be (30mm)\\
                \hline    
            \end{tabular}
        \caption{Information of Primary Beam}
        \label{tab:Primary_Beam}
    \end{table}
    The collision of the initial beam and the target created the secondary beam, including not only purpose isotope of this experiment, ${}^{19}$B, but also other neighboring isotopes. The main isotope used in this research is ${}^{17}$B, which needed to be separated and identified through the BigRIPS separator. 
    \begin{table}[ht]
        \centering
        \begin{tabular}[ht]{c|c|c}
            \hline
            isotope & Target & Average Energy at the middle of target \\
            \hline
            & C (1.789 g/cm$^2$)  & 270 MeV/u\\
            ${}^{17}$B & Empty  & 275 MeV/u\\
            & Pb (3.255 g/cm$^2$) & 270 MeV/u\\
            \hline    
        \end{tabular}
        \caption{Information of Secondary Beam}
    \end{table}
    The identified ${}^{17}$B was then transferred to SAMURAI, where it underwent Coulomb dissociation, the main objective of this research. Three different targets were used: C, Pb, and Empty. The details for ${}^{17}$B with each target were as follows.

\section{BigRIPS separator}

    \begin{figure}
        \centering
        \includegraphics[width=12cm]{chapter3/BigRIPS_roof.jpg}
        \caption{A top view of the BigRIPS separator \cite{Dayonewiki}}
        \label{fg:BigRIPS}
    \end{figure} 

After the primary ${}^{48}$Ca beam is accelerated at the last superconducting cyclotron SRC, the beam bombarded on a 30mm thick Be target and the secondary beam $^{17}$B is produced through the in-flight fragmentation method. Since the secondary beam includes many rare isotopes, the RI beam is separated by the BigRIPS separator. Table 3.1 shows the BigRIPS separator setup for Dayone experiment. In the first stage of BigRIPS separator, the RI beam separated by dipole magnet with slit and wedge-shaped degrader located at F1 focal plane. In the dipole magnet, the rigidity of a particle can be written as following eq (\ref{eq:rigidity}).
    \begin{align}
        B\rho = v \cdot \frac{A}{Z}  \label{eq:rigidity}
    \end{align}
The velocity of the secondary beam is almost same regardless of the nuclei so that it can be possible to choose specific $A/Z$ by adjusting the slit to filter only some $B\rho$ value. After that, the wedge-shaped degrader makes the beam energy depends on the $Z$ so that even same $A/Z$ nuclei can be separated. In the second stage, the RI beam is identified using TOF-B$\rho$-$\Delta E$ method. Each detector information will be described as following.

    \begin{table}[h]
        \centering
        \begin{tabular}{c c c c}
            \hline
            Location & Dipole [Tm] & Slit[mm] & Degrader / Detector \\
            \hline
            F0  &  &  & Be target (30mm) \\
            F1  &  & $\pm$120 & Al wedge degrader (15mm)\\
            F1-F2 &   8.985  &  & \\
            F2  &  & L: 10 R: 7 & \\
            F3  &  &  & Plastic Scintillator (3mm)\\
            F3-F4 & 8.954 &  & \\
            F4-F5 & 8.96 &  & \\
            F5  &   & $\pm$120 & BPC \\
            F5-F6 & 8.775  &  & \\
            F6-F7 & 8.785  &  & \\
            F7  &  & $\pm$120 & Plastic Scintillator (3mm)\\
            F8 &  & $\pm$170 & \\
            F12 &  & $\pm$170 & \\
            %F13 &  & & Plastic Scintillator (0.5mm$\times$2)\\
            \hline
        \end{tabular}
        \caption{BigRIPS separator setup for Dayone experiment \cite{Dayonewiki}}
    \end{table}

\subsection{Plastic Scintillator}
At focal plane F3, F7, F13, plastic scintillators are located for measuring the time of flight (TOF) of secondary beam. In F3 and F7, plastic scintillator with 3mm thickness is located and at F13 there are two scintillators, SBT1 and SBT2, with 0.5mm thickness. The flight length between F7 and F13 (Average point of SBTs) is 3662.00mm.

\begin{table}[h]
    \centering
    \begin{tabular}{c|cc}
        \hline
        Location & Thickness & Distance from target upstream \\
        \hline
        F3 & 3mm & 86053.56 mm\\
        F7 & 3mm & 39483.58 mm\\
        F13-1 &0.5mm & 2904.08 mm\\
        F13-2 &0.5mm & 2824.08 mm\\
        \hline
    \end{tabular}
    \caption{Information of Plastic Scintillators at F3, F7, F13}
\end{table}


\subsection{BPC (Beam Proportional Chamber)}
BPC is Multi Wire Proportional Chamber (MWPC) located at F5 focal plane which is used for measuring the position of beam. The purpose of BPC is tagging magnetic rigidity and momentum of secondary beam.
\begin{table}[h]
    \centering
    \begin{tabular}{l|c}
        \hline
        Effective Area & (H)240mm $\times$ (V)150mm \\
        Configuration & $XX$ (2 Planes) \\
        number of wire & 64 $\times$ 2 = 128 \\
        Wire Pitch & 4mm \\
        Gas & $i$--${\text{C}}_{4} {\text{H}}_{10}$ at 50 torr\\
        \hline
    \end{tabular}
    \caption{Parameter of BPC (Beam Proportional Chamber) \cite{SAMURAI}}
\end{table}


\begin{figure}[h]
    \centering
    \begin{subfigure}[h]{\textwidth}
        \centering
        \includegraphics[width=12cm]{chapter3/bpc_a1.jpg}
    \end{subfigure}
    \begin{subfigure}[h]{\textwidth}
        \hspace{2.4cm}
        \includegraphics[width=12.5cm]{chapter3/bpc_a23.jpg}
    \end{subfigure}
        \caption{Schematic View of BPC (Beam Proportional Chamber) \cite{SAMURAI}}
\end{figure}

\clearpage

\section{SAMURAI}

\begin{figure}[hbt!]
    \centering
    \includegraphics[width=12cm]{chapter3/SAMURAI1.png}
    \caption{A top view of the beam line from BigRIPS to SAMURAI spectrometer}
\end{figure}

The SAMURAI spectrometer is designed for kinematically complete experiment such as invariant mass spectroscopy. \cite{SAMURAIConcept} Charged fragment bent by SAMURAI superconducting magnet and detected by two drift chambers (FDC1, FDC2) and one plastic scintillator (HODF). Two drift chamber for fragment are located at before and after SAMURAI magnet, for rigidity analysis. And plastic scintillator HODF is placed after FDC2 to measure the TOF and energy loss of fragment. Finally the neutron detector array NEBULA is located at the end of extended beam line for neutron detection. Using SAMURAI system, the invariant mass of the system can be reconstructed by measuring all of the fragments and neutrons.

\begin{figure}[t]
    \centering
    \includegraphics[width=12cm]{chapter3/SAMURAI.png}
    \caption{A top view of the SAMURAI spectrometer}
\end{figure}


\subsection{ICB (Ion Chamber for Beam)}
The ICB is multi-layer ionization chamber for measuring the energy loss ($\Delta E$) of secondary beam. Using P10 gas at 1 atm, the energy loss of secondary beam can be measured. 
\begin{table}[h]
    \centering
    \begin{tabular}{l|c}
        \hline
        Effective Area & (H)140mm $\times$ (V)140mm $\times$ (D)420mm\\
        Configuration & 10 anodes and 11 cathodes \\
        Anode-cathode gap & 21mm \\
        Gas & P10 at 1 atm\\
        Distance from target upstream & 476.87 mm \\
        \hline
    \end{tabular}
    \caption{Parameter of ICB (Ion Chamber for Beam) \cite{SAMURAI}}
\end{table}

\begin{figure}[t]
    \centering
    \includegraphics[width=11cm]{chapter3/icb_a}
    \caption{Schematic View of ICB (Ion Chamber for Beam) \cite{SAMURAI}}
\end{figure}

\subsection{BDC1, BDC2 (Beam Drift Chamber)}
Before target, there are two Beam drift chamber for reconstructing the trajectory of secondary beam. Using the trajectory information, the position of secondary beam at target can be calculated. In this experiment, each BDC box is filled with $i$--${C}_{4} {H}_{10}$ gas at 100 torr. 

\begin{table}[h]
    \centering
    \begin{tabular}[h]{l|c}
        \hline
        Effective Area & (H)80mm $\times$ (V)80mm\\
        Configuration & $XX'YY'XX'YY'$ (8 planes)\\
        Number of Wire & 16 $\times$ 8 = 128 \\
        Wire Pitch & 5mm \\
        Gas & $i$--${\text{C}}_{4} {\text{H}}_{10}$ at 100 torr\\
        Distance from target upstream & (BDC1) 2032.12mm (BDC2) 1032.8 mm \\
        \hline
    \end{tabular}
    \caption{Parameter of BDC (Beam Drift Chamber) \cite{SAMURAI}}
\end{table}

\begin{figure}[h]
    \centering
    \begin{subfigure}{\textwidth}
        \centering
        \includegraphics[width=12.5cm]{chapter3/bdc_a1.jpg}    
    \end{subfigure}
    \begin{subfigure}{\textwidth}
        \hspace{1.5cm}
        \includegraphics[width=12cm]{chapter3/bdc_a3.jpg}
    \end{subfigure}
    \caption{Schematic View of BDC (Beam Drift Chamber) \cite{SAMURAI}}
\end{figure}

\clearpage

\subsection{SBV (Secondary Beam Veto)}

\subsection{Secondary Target}

\begin{table}[h]
    \centering
    \begin{tabular}{c|c}
        \hline
        Material & Thickness \\
        Pb & 3.255 g/cm${}^{2}$\\
         C & 1.789 g/cm${}^{2}$ \\
        \hline
    \end{tabular}
    \caption{Information of Secondary Reaction Target \cite{Dayonewiki}}
\end{table}

\subsection{DALI}

\subsection{SAMURAI Magnet}
Charged particle decayed from ${}^{17}$B is detected by the SAMURAI detector system. After reaction at secondary target, the secondary beam ${}^{17}$B is dissociated to charged particle ${}^{15}$B and two neutrons. The charged particle ${}^{15}$B bent by SAMURAI superconducting dipole magnet. Its trajectory is recorded by FDC1 and FDC2 before and after SAMURAI. And the energy loss and time of flight is measured by HODF plastic scintillator after FDC2.



\begin{table}[h]
    \centering 
    \begin{tabular}{l|c}
    \hline
    Type & Superconducting dipole magnet \\
    Magnet Pole & $\phi$ 2m (0.88 m gap) \\
    Maximum field & 3.1 T \\
    Maximum current & 563 A \\
    Acceptance & $\theta_H \leq \pm 10{}^{\circ}$, $\theta_V \leq \pm 5{}^{\circ}$\\
    \hline
    \end{tabular}
    \caption{Parameter of SAMURAI Magnet \cite{SAMURAI}}
\end{table}

\subsection{FDC1, FDC2 (Forward Drift Chamber)}
After target, there are two Forward Drift Chamber for reconstructing the trajectory of charged fragment. Using the trajectory information, the rigidity of charged fragment at target can be calculated. In this experiment, FDC1 is filled with $i$--${C}_{4} {H}_{10}$ gas at 50 torr and FDC2 is filled with He + 50\% ${C}_{2} {H}_{6}$ gas at 1 atm. 

\begin{table}[h]
    \centering
    \begin{tabular}{l|c}
        \hline
        Effective Area & (H)400mm x (V)300mm x (D)180mm\\
        Configuration & $XX'UU'VV'XX'UU'VV'XX'$ (14 planes)\\
        Number of Wire & 32 $\times$ 14 = 448 \\
        Wire Pitch & 10mm \\
        Gas & $i$--${C}_{4} {H}_{10}$ at 50 torr\\
        Distance from target upstream & 1151.38 mm  \\
        \hline
    \end{tabular}
    \caption{Parameter of FDC1 (Forward Drift Chamber 1) \cite{SAMURAI}}
\end{table}

\begin{table}[h]
    \centering
    \begin{tabular}{l|c}
        \hline
        Effective Area & (H)2296mm x (V)836mm x (D)860mm\\
        Configuration & $XX'UU'VV'XX'UU'VV'XX'$ (14 planes)\\
        Number of Wire & 112 $\times$ 14 = 1568 \\
        Wire Pitch & 20mm \\
        Gas & He + 50\% ${C}_{2} {H}_{6}$ at 1 atm\\
        \hline
    \end{tabular}
    \caption{Parameter of FDC2 (Forward Drift Chamber 2) \cite{SAMURAI}}
\end{table}

\subsection{HODF (HODoscope for Fragment)}
The plastic scintillator HODscope is located behind FDC2 for measuring the TOF and energy loss of charged fragment. For the charged fragment identification, TOF-B$\rho$-$\Delta E$ method is used as same as beam particle identification at BigRIPS. For 
\begin{table}[h]
    \centering
    \begin{tabular}{l|c}
        \hline
        Effective Area & (H)1600mm x (V)1200mm x (D)10mm\\
        Number of Scintillator & 16 \\
        Width of Scintillator & 10mm \\
        \hline
    \end{tabular}
    \caption{Parameter of HODF (HODscope for Fragment) \cite{SAMURAI}}
\end{table}

\subsection{NEBULA}
For measuring momentum vector of neutron, 

\subsection{Geometry Information of SAMURAI Setup}

\begin{figure}
    \centering
    \includegraphics[width=\textwidth]{chapter3/minakata-geometry-drawing_20131121.pdf}
    \caption{Geometry information of SAMURAI setup}
\end{figure}

\section{Run summary}
\begin{center}
    \begin{tabular}[h]{c|ccc}
        \hline
        Run number& Target & Trigger & Note\\
        \hline
        394 - 404 &  C (1.789 g/cm${}^{2}$)  & DSB(1/20) $\cup$  B x N + D(1/1) &\\
        405 - 409 &  Empty  & DSB(1/20) + B x N + D(1/1) &\\
        410 - 427 &  Pb (3.255 g/cm${}^{2}$)  & DSB(1/20) + B x N + D(1/1) &\\
        428, 429, 431 & Pb (3.255 g/cm${}^{2}$)  & DSB(1/1) & F5 slit $\pm$1mm \\
        430 & Pb (3.255 g/cm${}^{2}$)  & DSB(1/1) & F5 slit $\pm$5mm \\
        \hline
    \end{tabular}
\end{center}

\section{Electronics}
\subsection{Data Acquisition System and Trigger condition}
There are four trigger conditions used in this experiment; DSB, B $\cap$ N, B $\cap$ N, B $\cap$ N. They are defined as follows.
\begin{enumerate}
    \item \textbf{DSB} (Down Scale Beam) 
    \item \textbf{B $\cap$ N} (Coincidence between Beam and NEBULA)
    \item \textbf{B $\cap$ D} (Coincidence between Beam and DALI)
    \item \textbf{B $\cap$ H} (Coincidence between Beam and HODF)
\end{enumerate} 

\begin{figure}[h]
    \centering
    \hspace{1cm}
    \includegraphics[width=12cm]{chapter3/Beam_Trigger.png}
    \caption{Logic Diagram for Beam Trigger}
\end{figure}

\begin{figure}
    \centering
    \includegraphics[width=12cm]{chapter3/NEB_Trigger.png}
    \caption{Logic Diagram for Beam Trigger}    
\end{figure}

\begin{figure}
    \centering
    \includegraphics[width=8cm]{chapter3/DALI_Trigger.png}
    \caption{Logic Diagram for Beam Trigger}
\end{figure}

\subsection{Live Time}
The DAQ readout rate is limited by the dead time of the DAQ sub-system. And the dead time depends on the trigger condition and the target.  
\begin{table}[h]
    \centering
    \begin{tabular}{c|c|cc}
        \hline
        Run number & target & DSB & B $\cap$ N, B $\cap$ D \\
        \hline
        394 - 404 & C & 0.843 & 0.815 \\
        405 - 409 & Empty & 0.890 & 0.854 \\
        410 - 427 & Pb & 0.861 & 0.831 \\
        \hline
    \end{tabular}
    \caption{Live time of each reaction trigger}
\end{table}
