\chapter{Data Analysis}
In the analysis of experimental data, the primary goals are to extract the 2n removal cross section in the ${}^{17}B \to {}^{15}B + 2n$ reaction and to obtain the Relative energy spectrum. To achieve this, I will describe the procedure of identifying the secondary beam of ${}^{17}B$, and \textcolor{red}{selecting events involving the fragment ${}^{15}B$ and two neutrons. Particular emphasis is placed on the process of selecting two-neutron events and the rejection of cross-talk.} The flow of the Data Analysis is as follows.

\begin{center}
    \begin{enumerate}
        \item Select the event contaning ${}^{17}B$ beam by beam PID
        \item Select the event contaning ${}^{15}B$ fragment by fragment PID
        \item Select the event contating two neutron by cross-talk analysis
        \item Extract the 2n removal cross section of the ${}^{17}B \to {}^{15}B + 2n$ reaction
        \item Reconstruct Invariant Mass at target and obtain the Relative energy spectrum 
    \end{enumerate}
\end{center}

\section{Analysis of the secondary beam}
The primary beam is generated by SRC, RIBF. By accelerating ${}^{48}$Ca in  MeV/u, bombarded on thick Be target (cm) secondary beam is produced. 초기 빔의 정보는 다음과 같다.
\begin{center}
    \begin{tabular}[h]{c|c|c}
        \hline
        Primary Beam & Beam Energy & Primary Target  \\
        \hline
        ${}^{48}Ca$ & 278.1 MeV/u & Be (3cm)\\
        \hline    
    \end{tabular}
\end{center}

본 실험에서 사용할 메인 핵자는 17B로, 초기 Ca 빔과  Be 타겟의 충돌로 만들어진다. 17B는 BigRIPS sarparator를 통해 분리된고, SAMURAI로 전달된다. 본 실험에서는 3가지 타겟 C, Pb, Empty를 사용하였다. 각각 타겟에 대한 B17의 정보는 다음과 같다.

\begin{center}
    \begin{tabular}[h]{c|c|c}
        \hline
        Secondary Beam & Target & Energy at Target \\
        \hline
        ${}^{17}B$ & C () & \\
        ${}^{17}B$ & Empty & \\
        ${}^{17}B$ & Pb () & \\
        \hline    
    \end{tabular}
\end{center}

Identification of ${}^{17}$B secondary beam  is performed by TOF-B$\rho$-$\Delta$E method. 
\subsection{Time of Fligh and $\beta_{Beam}$}
\subsection*{Time of Flight}
The time-of-filght (TOF)는 두 plastic scintillator사이의 검출 타이밍 차를 이용하여 계산한다. 이용되는 plastic scintilator는 F7에 위치한 신틸레이터와 F13에 위치한 SBT1, SBT2라 불리는 두 신틸레이터이다. TOF between F7 and F13 는 다음과 같이 정의된다.
\begin{displaymath}
    TOF_{F7-F13} = \frac{t_{SBT1} + t_{SBT2}}{2} - t_{F7} + \Delta t_{offset}
\end{displaymath}

$\Delta t_{offset}$ 실제 측정된 $TOF_{F7_F13}$과 Energy loss 계산으로 구해진 $TOF_{F7_F13}$ 의 차이를 보정하기 위한 offset이다. 이는 f5 slit을 +/- 1mm로 매우 좁힌 run 428,429,431를 이용하여 구한 TOF와 Energy loss 계산으로 구한 TOF의 차이를 이용하여 구한다. 계산으로 얻어진 $TOF_{F7_F13}$은 192.333ns 7를 보정하면 다음과 같다.\\
또한 두 신틸레이터의 시간정보에서 상관에 크게 벗어나는 이벤트를 제거함으로써 불필요한 노이즈를 제거할 수 있다. \\
\indent TOF713을 특정했으면, 이를 이용하여 Beam의 $\beta$를 다음과 같이 구한다.
\begin{displaymath}
    \beta_{Beam} = \frac{dist_{F7_F13}}{TOF_{F7_F13} \times c}
\end{displaymath}
1.723e-05 x - 0.009899 x + 1.902
\subsection{Magnetic Rigidity $B\rho$}
Magnetic Rigidity $B\rho$는 F5에 위치한 Beam Projection Chamber (BPC)를 이용하여 도출한다. 
BPC에서는 F5 Focal point를 중심으로 입자의 x좌표를 구한다. 
\begin{displaymath}
    B\rho = (1+\frac{x}{D}) B\rho_{0}
\end{displaymath}
D = 3300 (mm) Dispersion, $B\rho_{0}$ = 8.775 (Tm) \\
여기서 A/Z를 도출하기 위해서는 위에서 구한 $\beta_{F7_F13}$를 F5에서의 $\beta_{F5}$로 보정할 필요가 있다. \\
\indent 이렇게 A/Z를 특정지을 수 있다.
\begin{displaymath}
    \frac{A}{Z} = \frac{B\rho_{F5}}{B\rho_{F5}^{ref}} \times \frac{\beta_{F5}^{ref}}{\beta_{F5}}
\end{displaymath}

\subsection{Energy Loss $\Delta$E}
각 입자의 에너지 손실 $\Delta$E는 BigRIPS 하류에 놓여있는 Ionize Chamber for Beam (ICB)에서 측정된다. $\Delta$E와 Z사이의 상관은 Bethe-Bloch의 Energy loss 공식에 따라 다음과 같이 구할 수 있다.
\begin{displaymath}
    \frac{dE}{dx} = \frac{4\pi N_{A} r_{e}^{2} m_{e} c^{2} z^{2}}{\beta^{2}} \left[ \frac{1}{2} \ln \frac{2m_{e}c^{2}\beta^{2}\gamma^{2}T_{max}}{I^{2}} - \beta^{2} - \frac{\delta(\beta\gamma)}{2} \right]
\end{displaymath}

\subsection{Beam Particle Indeification}
위의 과정에서 구한 TOF, B$\rho$, $\Delta$E 를 통해 Z와 AoZ를 다음과 같이 구한다. \\

다음은 Beam Particle Identification의 결과이다. 

\section{Beam Profile at Target}
\subsection{BDC Profile}
\subsection{Target Profile}
Target Chamber 주변에는 검출기가 없기 때문에 두개의 BDC를 이용하여 target에서의 Beam 위치를 외삽한다. 본 실험에서 사용하는 ${}^{17}B$는 본 실험의 main beam이 아니기 때문에, 위에서 확인한 BDC에서의 beam profile 과 같이 target 중심에서 ずれている。따라서 Target의 유효면적 x +/- 35mm, y +/- 35mm를 정의하고, 이를 넘어가는 이벤트는 제거한다.

\subsection{Beam beta calculation}
Invariant Mass를 계산하기 위해서는 Beam의 beta를 계산해야 한다. Beam의 beta는 다음과 같이 계산된다.


\section{Analysis of charged fragment}
\subsection{FDC를 이용한 위치 계산}
\subsection{Brho from Geant4 simulation}
\subsection{HODscope Z}
15B를 특정하기 위하여, Hodscope라는 Plastic Scintillator 검출기에서 Z를 특정한다.
상류에서 17B를 특정하여, 가장 많은 입자가 Z=5라고 가정, 위의 시뮬레이션으로 얻은 AOZ를 대입하여 15B를 특정한다.
\subsection{Fragment Particle Identification}

\clearpage

\section{Analysis of Neutrons}
2차 빔 ${}^{17}$B에서 방출된 중성자는 사무라이 시스템의 중성자 검출기 NEBULA를 통해 검출된다. 본 섹션에서는 중성자 검출기 NEBULA를 이용하여 1중성자 이벤트 혹은 2중성자 이벤트를 선택하는 과정에 대해 서술한다. 중성자의 운동량 모멘텀 $P(n)$은 타겟으로부터의 NEBULA에서의 검출 위치, 그리고 비행시간 TOF로 Reconstructed. 중성자의 은 다음과 같이 기술된다.


\section{Cross-talk Rejection}
(중성자의 검출 방법 서술) 두 중성자의 이벤트를 추출하는 과정에서 가장 중요한 것이 Cross-talk 제거이다. Cross-talk란, 하나의 중성자가 여러개의 신호를 만드는 것으로, 두 중성자 이벤트를 선택할 때에 있어 가장 많은 노이즈를 차지한다. 
Cross talk rejection을 위해 Geant4 시뮬레이션을 통해 ${}^{16}B\to{}^{15}B+n$ 이벤트를 생성시켜, 모든 2중성자 이벤트가 Cross-talk인 경우를 재현하였다. 실행한 Geant4 시뮬레이션 정보는 다음과 같다.
\begin{center}
    \begin{tabular}[h]{c|c}
        \hline
        Beam Energy & 140 MeV/u\\
        Relative Energy & 0.5 MeV/u\\
        \hline
    \end{tabular}
\end{center}


위의 시뮬레이션으로 구한 Cross-talk 조건을 적용하였을 때, Cross-talk이벤트가 남아있는 비율을 평가하였다. 각 조건을 단계 a,b,c,d로 

Cross-talk 제거에는 크게 3가지 단계가 있다. 
이하 시뮬레이션의 결과로 결정한 각 단계의 Cross-talk를 실제 실험 데이터에 적용하여 제거한 결과를 서술한다.
\subsection{gamma event rejection}
진짜 뉴트론 이벤트를 선택하기 위해서 감마선을 제거하는 스레숄드를 지정한다.
\begin{enumerate}
    \item 1st VETO에 hit한 이벤트는 모두 하전입자라 간주하여 제거한다.
    \item 중성자 검출기 NEBULA에 입사한 이벤트 중, 발광량 Q가 6MeVee 이하인 이벤트는 감마선으로 간주하여 제거한다. 또한 하나의 플라스틱 신틸레이터 모듈을 통과할 때의 중성자의 최대 에너지 loss 130MeV를 넘어가는 이벤트도 중성자에서 산란된 다른 이벤트로 간주하여 제거한다.
    \item Target에서부터의 TOF가 1st wall의 경우 40ns 이하, 2nd wall의 경우 42ns 이하인 이벤트 또한 중성자 이외의 이벤트로 간주하여 제거한다.
    \item (2중성자 이벤트에 대해서만) 2nd VETO에 hit한 이벤트에 대하여, 2nd NEUT wall에 입사한 가장 빠른 두 중성자 이벤트가 dr(xy)<500mm, 2ns <dt< 5ns 인 이벤트는 2nd VETO에서 기인한 cluster 산란 이벤트로 간주하여 제거한다.
\end{enumerate}
\subsection{Clustering Event Subtraction}

\subsection{same wall event}
두 중성자

\subsubsection{cluster proton cross-talk}


\subsection{different wall}
\section{Cross-talk 잔존률 평가}

\section{Acceptance and Efficiency}

NEBULA의 Acceptance and Efficiency를 평가하기 위해 Geant4 Simulation을 진행했다. 