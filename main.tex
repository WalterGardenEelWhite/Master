\documentclass{book}[12pt]
% !TEX encoding = UTF-8 Unicode
\usepackage{dhucs}
\usepackage{dhucs-trivcj}
\usepackage{geometry}
\geometry{left=30mm,right=30mm,top=30mm,bottom=30mm}

\usepackage[dvipdfmx]{graphicx}
\graphicspath{{./Figure/}}
\usepackage[nottoc]{tocbibind}
\usepackage{amsmath,amssymb}
\usepackage{enumitem}%좀 더 fancy한 itemize를 위한 패키지
\usepackage{changepage}%문단을 들여쓰기 하기 위한 패키지
\usepackage{subcaption}%sub figure를 위한 패키지
\usepackage{array}%표를 그리기 위한 패키지
\renewcommand{\arraystretch}{1.2} % 행 간 간격을 1.5배로 설정
\setlength{\tabcolsep}{8pt} % 열 간 간격을 10pt로 설정
\usepackage[sorting=none, style=numeric]{biblatex}
%\bibliographystyle{unsrt}
\bibliography{reference}
%\addbibresource{reference.bib}

\usepackage{hyperref}
\usepackage{pict2e} % for drawing picture

\usepackage{here}
\usepackage{url}
\usepackage{mathrsfs}
\usepackage[version=3]{mhchem}
\usepackage{indentfirst}\setlength\parindent{2em}
\usepackage{color}


\usepackage{kotex}

%\newenvironment{m}{\begin{eqnarray}}{\end{eqnarray}}
%\setcounter{tocdepth}{2}

%\def\theequation{\arabic{chapter}.\arabic{section}.\arabic{equation}}
%\makeatletter
%\@addtoreset{equation}{section}
%\makeatother

%\def\thline{\noalign{\hrule height 1pt}}
%\def\tvline{\vrule width 1pt} 

%\renewcommand{\appendixname}{Appendix}
%\renewcommand{\bibname}{$B;29MJ88%(B}
%\setlength{\textwidth}{\fullwidth}
%\setlength{\evensidemargin}{\oddsidemargin}



\title{Coulomb Dissociation of Borromean Nuclei ${}^{17}B$}
\author{Hyeji Lee}
\date{\today}

\begin{document}
\frontmatter
\maketitle
\pagenumbering{arabic}
%\pagenumbering{romanw}
\clearpage

\thispagestyle{empty}

\clearpage

\thispagestyle{empty}
\chapter*{Abstract}
Recent studies on neutron-rich isotopes have highlighted their distinctive two- neutron (2n) halo structure, where halo neutrons are weakly bound and spatially extended from the core. This research studies the 2n halo nucleus 17B, which can also be the core of 19B, to understand its implications for multi-neutron halo structures and dineutron correlation. Using the SAMURAI spectrometer at RIBF, RIKEN, a 48Ca primary beam on a Be target produced the 17B secondary beam. This beam was dissociated into 15B and two neutrons on a Pb target. The 15B fragment and the neutrons were detected by the heavy-ion detectors and neutron detector array NEBULA, respectively, at SAMURAI, RIBF. We successfully extracted the two-neutron removal cross sections 733±8 (stat.) mb for lead target and 191±3 (stat.) mb for carbon target and constructed the relative energy spectrum between 15B and two neutrons using the invariant mass method. This result yielded a spectrum of the reduced transition probability for electric dipole, B(E1) distribution, with the photon equivalent method and Coulomb Dissociation cross sections spectrum. The obtained B(E1) spectrum has wider width and smaller strength compared to that of 19B, indicating the hindered halo structure of 17B

\clearpage

\tableofcontents
\listoffigures
\listoftables


\mainmatter
\pagenumbering{arabic}

\chapter{Introduction}
Over recent years, The study of neutron-rich nuclei has increasing attention in the field of nuclear physics, providing invaluable insights into the conventional nuclear models and interactions in the condition far from stable nuclei. One particularly intriguing subclass is neutron halo nuclei\cite{Tanihata96}, which feature an extended halo of one or two loosely bound neutrons far from its core. The first discovery of neutron halo nuclei was ${}^{11}$Li by I. Tanihata et al\cite{Tanihata85}. In Figure 1.1, ${}^{11}$Li has enormously large radius compared to neighboring  He, Li, Be and C isotopes which has similar A number. Based on this discovery, he suggested that ${}^{11}$Li has a large deformation or a long tail in matter distribution. After that, P. G. Hansen and B. Jonson\cite{HansenandJonson} suggested ${}^{11}$Li as a neutron halo nuclei due to the large neutron radius compared to its core ${}^{9}$Li.

\begin{figure}
    \centering
    \includegraphics[width=7cm]{chapter1/Radius_of_11Li.png}
    \caption{Matter rms radius of He, Li, Be and C isotopes. ${}^{11}$Li has significantly large radius compared to other Li isotopes.}
\end{figure}

After more research on this theme, many people suggested how to decide a nuclei as a halo nuclei. One is obviously a small neutron separation energy. Compared to common separation energy of stable nuclei 6 - 8MeV, many neutron rich nuclei have extremely small separation energy less than 1 MeV. And because of this, the halo nuclei usually have large radius. Given that the neutron wavefunction is a square well potential, the wavefunction of neutron can be written as
\begin{align}
    \Psi(r) = \bigg(\frac{2\pi}{k}\bigg)\bigg(\frac{-e^{k r}}{r}\bigg)\bigg[\frac{e^{k R}}{(1+ k R)^{1/2}}\bigg],
\end{align}
where R is the width of the potential. With this wavefunction, the density distribution of the neutron is
\begin{align}
    \rho(r) = |\Psi(r)|^2 \propto \frac{e^{-2k r}}{r^2}.
\end{align}
The factor $k$ determines the tail of the neutron, and it is related to the neutron separation energy by
\begin{align}
    (\hbar k)^2 = 2\mu S_n,
\end{align}
where $\mu$ is the effective mass of system and $S_n$ is the neutron separation energy. From this equation, we can see that the $k$ is inversely proportional to the neutron separation energy. So, the smaller the neutron separation energy is, the larger the $k$ is. And the larger the $k$ is, the longer tail of neutron is. So, the halo nuclei usually have small neutron separation energy and large radius.

% halo feature 2
The narrow momentum distribution is also the feature of halo nuclei. The momentum distribution of neutron [$f (p)$] can be expressed by the Fourier transform of the wavefunction (1.1),
\begin{align}
    f(p) = \frac{C}{p^2+(\hbar k)^2} %C / (p^2 + k^2), 
\end{align}
where $p$ is the momentum of the neutron and the width of the distribution is defined by $k$ which related with separation energy. In this formula, the concept of Heigenberg uncertainty principle is shown. The smaller the $k$, the wider the density distribution, and also the narrower the momentum distribution is. 

% halo feature 3
Another parameter which define a halo nuclei is the orbital angular momentum of valance neutron. The shape of momentum distribution depends on not only the separation energy, but also the angular momentum ($l$) of the orbital. The width of The centrifugal barrier which proportional to $l(l+1)/r^2$ largely affect the density distribution of neutron. 
%Also in three-body halo system, the centrifugal barrier is defined in the form of $[(K+3/2)(K+5/2)]/\rho^2$ by Fedorov et al.\cite{fedorov} with the quantum number $K$ (usually called hypermomenent). In this formula, the hight of centrifugal barrier rapidly increases with $K$. Ad even $K = 0$ has centrifugal barr

low orbital angular momentum of valance neutron makes the halo nuclei.

% halo feature 4
The most important feature of a halo nucleus focused on this research is soft mode of electric dipole excitation. Given that halo nuclei has a core and loosely bounded valance neutron, the core and valance neutron can be polarized by the external electric field. And this dipole excitation mode is shown in very low energy region, because the halo structure response easily to the external electric field. This soft mode of electric dipole excitation is called Soft E1 excitation. This feature predicted by Hansen and Jonse\cite{HansenandJonson}, and also Ikeda et al.\cite{Ikeda}.

In E1 reduced transition probability, the term 'reduced' means that the transition matrix element is independent of the magnetic substate of the initial and final states. The E1 reduced transition probability is defined as
\begin{align}
    B(E1) = \frac{|\langle \Phi_f ||\hat{T}(E1)|| \Phi_i\rangle|^2}{(2J_i + 1)},
\end{align}

In boron isotopes, the soft E1 excitation of ${}^{19}$B is observed by K. J. Cook et al.\cite{KJCook} using Coulomb dissociation method. (result briefly)


\begin{figure}[t]
    \centering
    \includegraphics[width=10cm]{chapter1/19B_B(E1).png}
    \caption{The B(E1) distribution of ${}^{19}$B. The peak at 0.5 MeV is the soft E1 excitation.}
\end{figure}

But how to define halo nuclei is still in debate. Because there is no clear definition what separation energy halo nuclei should have, or what radius halo nuclei should have. And the mixture between several orbital angular momentum of valance neutron makes the situation even more difficult to interpret. For example, 

One example is two neutron halo nuclei in boron isotope, ${}^{17}$B and ${}^{19}$B. The first study about these nuclei was performed by T. Suzuki et al\cite{suzuki99}. They measure the reaction cross section of boron isotope and extracted the matter radius of them. 

\begin{figure}
        \centering
        %\includegraphics[width=14cm]{chapter1/Radius_of_boron.png}
    \end{figure}

But there were many discussion between the structures of 2-neutron halo nuclei of Boron isotopes. T. Suzuki et al. assumed that both ${}^{17}B$ and ${}^{19}$B is 2-neutron halo nuclei, due to the large radius of both ${}^{17}$B and ${}^{19}$B. But he also suggested ${}^{19}$B is hindered 2-neutron halo nuclei due to the small ratio between halo and core radius.


\begin{align}
    \Psi({}^{19}\text{B}) = \psi({}^{17}\text{B}) \otimes [\alpha |1d_{5/2} \rangle^2 + \beta |2s_{1/2} \rangle^2]    
\end{align}

\begin{figure}
    \centering
    \includegraphics[width=10cm]{chapter1/17B_ZHYang.png}
\end{figure}
But recently, Z. H. Yang et al. \cite{ZHYang} shows a strong evidence that ${}^{17}$B might not be a two neutron halo nuclei. He performed the quasi-free ($p,pn$) scattering reaction and extracted the relative energy distribution of ${}^{16}$B. He assumed that the knocked-out neutron from the ${}^{17}$B  should almost come from $1d_{5/2}$ or $2s_{1/2}$ orbital and got the relative energy distribution of ${}^{16}$B. (Figure 1.3) 
And also, he obtained the spectroscopic factors for $1d_{5/2}$ and $2s_{1/2}$ orbital and there was surprisingly small percentage 9(2)\% of $2s_{1/2}$ orbital. This result is very different from the ones, which are 36(19)\%, 69(20)\%, 50(10)\% and by T. Suzuki and Y. Yamaguchi.

Also Estradé et al. \cite{Estrade} suggested there is a
\\
So, in this research, we study the Soft E1 excitation of ${}^{17}$B, which can be the final key for deciding ${}^{17}$B as a halo nuclei or not. In this thesis, we will discuss the Coulomb Dissociation of ${}^{17}$B as a tool for searching the Soft E1 excitation of ${}^{17}$B. 

\begin{align}
    \Psi({}^{17}\text{B}) = \psi({}^{15}\text{B}) \otimes [\alpha |1d_{5/2} \rangle^2 + \beta |2s_{1/2} \rangle^2]
\end{align}

\begin{align}
    \langle \text{cos} \theta_{nn} \rangle &= \langle \Psi({}^{11}\text{Li}) | \text{cos} \theta_{nn} | \Psi({}^{11}\text{Li}) \rangle \notag \\
    &= \alpha^2 \langle (1d_{5/2})^2 | \text{cos} \theta_{nn} | (1d_{5/2})^2 \rangle + \beta^2 \langle (2s_{1/2})^2 | \text{cos} \theta_{nn} | (2s_{1/2})^2 \rangle \notag \\
    &\hspace{4mm} + 2\alpha \beta \langle (1d_{5/2})^2 | \text{cos} \theta_{nn} | (2s_{1/2})^2 \rangle \notag \\
    &= 2\alpha \beta \langle (1d_{5/2})^2 | \text{cos} \theta_{nn} | (2s_{1/2})^2 \rangle
\end{align}

where $|(1d_{5/2}^2)$ represents $|\psi({}^{15}\text{B})\otimes|$

\chapter{Methods}

In this chapter, the methods used in this research are explained. First, Coulomb dissociation is introduced as a method to research the halo structure of $^{17}$B. For the part of it, equivalent photon method as a tool for investigating the soft $E1$ excitation of $^{17}$B is described. The $E1$ reduced transition probability $B(E1)$ and the geometrical information which is related to dineutron correlation can be obtained. Next, for extracting the Coulomb dissociation cross section from experimental data, the subtraction of nuclear breakup component is explained. Finally, the invariant mass method for the reconstruction of experimental data is explained. 

\section{Coulomb Dissociation}
\begin{figure}[t]
    \centering
    \setlength{\unitlength}{1mm}
    \begin{picture}(100,52)
        % Boron-17 nucleus
        \put(10,40){\circle{15}}
        \put(10,42){\circle{10}}
        \put(7,41){\footnotesize ${}^{15}$B}
        \put(8,35){\circle*{2}}
        \put(12,35){\circle*{2}}
        \put(8,30){\footnotesize n     n}
        \put(6,49){${}^{17}$B}

        % Arrow to excited state
        \put(20,40){\vector(1,0){20}}

        % Excited Boron-17 nucleus
        \put(50,40){\circle{15}}
        \put(50,42){\circle{10}}
        \put(47,41){\footnotesize ${}^{15}$B}
        \put(48,35){\circle*{2}}
        \put(52,35){\circle*{2}}
        \put(48,30){\footnotesize n      n}
        %\put(50,20){\line(0,-1){12}}
        \put(50,15){\circle{12}}
        \put(48,14){\footnotesize Pb}
        \put(46,49){${}^{17}$B$^*$}

        % Gamma ray
        \multiput(50,21)(0,1){7}{\line(0,1){0.5}}
        \multiput(50,9)(0,-1){7}{\line(0,-1){0.5}}
        \multiput(53,20)(0.4,1){7}{\line(0.4,1){0.2}}
        \multiput(47,20)(-0.4,1){7}{\line(-0.4,1){0.2}}
        \multiput(53,10)(0.4,-1){7}{\line(0.4,-1){0.2}}
        \multiput(47,10)(-0.4,-1){7}{\line(-0.4,-1){0.2}}

        %\multiput(50,15)(-3,3){5}{\line(0,-0.3){1}}
        \put(42,20){\footnotesize $\gamma$}

        % Arrow to final state
        %\put(60,20){\vector(1,0){20}}

        % Final state boron-15
        \put(90,47){\circle{10}}
        \put(87,45.5){\footnotesize ${}^{15}$B}
        \put(60,40){\vector(1,0.3){20}}
        \put(96,49){\footnotesize \( E_{^{15}\text{B}}, \vec{P}_{^{15}\text{B}} \)}

        % Final state neutron 1
        \put(90,33){\circle*{2}}
        \put(86,33.5){\footnotesize $n$}
        \put(60,40){\vector(1,-0.25){25}}
        \put(92,33){\footnotesize \( E_{n1}, \vec{P}_{n1} \)}
        % Final state neutron 2
        \put(90,28.5){\circle*{2}}
        \put(86,29){\footnotesize $n$}
        \put(60,40){\vector(1,-0.37){25}}
        \put(92,28){\footnotesize \( E_{n2}, \vec{P}_{n2} \)}

    \end{picture}
   \caption[The scheme of Coulomb dissociation]{The scheme of Coulomb dissociation of ${}^{17}$B. The ${}^{17}$B is induced to Pb target and excited by virtual photon made from electric magnetic field by relativistic movement between ${}^{17}$B and Pb target. The excited ${}^{17}$B is dissociated into ${}^{15}$B and two neutrons. The $E$ and $\vec{P}$ are represent total energy and momentum of each fragment respectively.}
   \label{fig:CD}
\end{figure}

Coulomb dissociation is breakup reaction from excited stated by Coulomb excitation. Figure \ref{fig:CD} shows the scheme of the Coulomb dissociation of ${}^{17}$B used in this research. When nuclei incident into the high-Z target like lead, the projectile is excited by the electric field of target. When the final state is above the decay threshold, the Coulomb dissociation occurs.

\subsection{Equivalent Photon Method}
Equivalent photon method\cite{Jackson}\cite{Bertulani}\cite{Aumann} is a powerful tool for investigating Coulomb excitation/dissociation in terms of the virtual photon. Under the equivalent photon method, Coulomb dissociation cross section $\sigma_{CD}$ can be described with the photon absorption cross section $\sigma_{\gamma}^{E1}(E_x)$ and the virtual photon number $N_{E1}(E_x)$ as
\begin{align}
    \frac{d\sigma_{CD}}{dE_x} = \frac{N_{E1}(E_x)}{E_x} \sigma_{\gamma}^{E1}(Ex),
\end{align}
where $E_x$ is the excitation energy of nuclei, $N_{E1}(E_x)$ is the virtual photon number produced by $E1$ transition. And the photon absorption cross section $\sigma_{\gamma}^{E1}$ can directly be related to the $E1$ reduced transition probability $dB(E1)/dE_x$ as
\begin{align}
    \sigma_{\gamma}^{E1} = \frac{16 \pi^3}{9 \hbar c} E_x \frac{dB(E1)}{dE_x}. \label{eq:photon_absorption}
\end{align}
Then, the Coulomb dissociation cross section is written as
\begin{align}
    \frac{d\sigma_{CD}}{dE_x} = \frac{16 \pi^3}{9 \hbar c} N_{E1}(E_x) \frac{dB(E1)}{dE_x}.
\end{align}
The virtual photon number for $E1$ transition $N_{E1}$ is obtained by investigating the photon flux at an impact parameter $b$ as
\begin{align}
    N_{E1}(E_x) &= \int_{b}^{\infty} 2\pi b n_{E1}(E_x, b) db  \\
                &=\frac{2}{\pi}Z^{2}_{1}\alpha\Big(\frac{c}{v}\Big)^{2}\Big[\xi K_{0}(\xi)K_{1}(\xi)-\frac{v^{2}\xi^{2}}{2c^{2}}(K^{2}_{1}(\xi)-K^{2}_{0}(\xi)\Big]
\end{align}

\begin{adjustwidth}{1cm}{1cm}
    $\xi = E_x b / \gamma v \hbar$ \\
    $E_{\gamma} = \omega \hbar$ : Virtual photon energy\\ 
    $Z_{1}$ : Atomic number of target\\
    $b$ : Impact parameter 1.3 ($17^{1/3} + 208^{1/3}$) = 11.048 fm\\
    $K_0, K_1$ : Modified Bessel function of order zero and one \\
    $\alpha = e^2 / \hbar c$ : Fine structure constants\\
\end{adjustwidth}
%\vspace{1mm}
In this experiment, we assumed the virtual photon energy $E_{\gamma}$ is equal to excitation energy $E_x$ of projectile. In Figure \ref{fig:Virtual_Photon}, the virtual photon number $N_{E1}(E_x)$ for $E1$ transition in the function of excitation energy $E_x$ and the reduced $E1$ transition probability $dB(E1)/dE_x$ corresponding to energy region are shown. You can see there are two peaks corresponding to soft $E1$ excitation and giant dipole resonance respectively. Since the virtual photon number is exponentially decreasing with the excitation energy, the equivalent photon method is best suited for investigating the excitation in low energy region. 

\begin{figure}[t]
    \centering
    \includegraphics[width=8cm]{chapter2/Virtual_Photon.png}
    \caption[Virtual photon number $N_{E1}(E_x)$ spectra and $dB(E1)/dE_x$ spectrum for halo nucleus]{Virtual photon number $N_{E1}(E_x)$ spectra with the $dB(E1)/dE_x$ spectrum for halo nucleus\cite{Nakamura23}. The peak near 1 MeV shows the soft $E1$ excitation, while the peak in high energy region ($\sim 20$ MeV) represents the giant dipole resonance.}
    \label{fig:Virtual_Photon}
\end{figure}

\subsection{Geometry of two neutron halo and dineutron correlation}
Another information which can be extracted from $E1$ reduced transition probability is related with the geometrical value of the two neutron halo nucleus. By non-energy weighted cluster sum rule from Esbensen et at. \cite{Esbensen}, the $E1$ reduced transition probability $B(E1)$ in entire energy region can be written as 
\begin{align}
    B(E1) &= \int_{-\infty}^{\infty} \frac{dB(E1)}{dE_x}dE_x \notag \\
        &= \frac{3}{\pi} \bigg(\frac{Z e}{A}\bigg)^2 \langle r^2_{c-nn} \rangle,
        %&= \frac{3}{4 \pi} \bigg(\frac{Z e}{A}\bigg)^2 \langle \vec{r_1}^2 + \vec{r_2}^2 + 2 \vec{r_1} \cdot \vec{r_2} \rangle \notag \\
        %&= \frac{3}{4 \pi} \bigg(\frac{Z e}{A}\bigg)^2 \langle \vec{r_1}^2 + \vec{r_2}^2 + 2\vec{r_1} \vec{r_2} \cos \theta_{nn} \rangle
\end{align}
where $\langle r^2_{c-nn} \rangle$ is a root mean square (rms) distance between the core nucleus and the two center of mass of two neutron. Furthermore, the distance between two neutrons $\langle r^2_{nn} \rangle$ can be obtained from matter radius of halo and core nucleus in the three body model as \cite{Bertulani07}\cite{Hagino07},
\begin{align}
    \langle r^2_{m} \rangle = \frac{A_c}{A} \langle r^2_{m} \rangle_{c} + \frac{2A_c}{A^2} \langle r^2_{c-nn} \rangle + \frac{1}{2A} \langle r^2_{nn} \rangle
\end{align}
where $A$ and $A_c (= A - 2)$ are the mass number of halo and core nucleus respectively. $\langle r^2_{m} \rangle$ and $\langle r^2_{m} \rangle_{c}$ are the matter radius of halo and core nucleus respectively. In this research, for calculation, we use the value $\langle r^2_{m} \rangle = 3.00 (6)$ fm, $\langle r^2_{m} \rangle_c = 2.75 (6)$ fm from the rms radius of $^{17}$B and $^{15}$B respectively \cite{Estrade}. $\langle r^2_{nn} \rangle$ is the distance between two neutrons.
Finally, the opening angle between two neutrons $\langle \theta_{nn} \rangle$ can be obtained as,
\begin{align}
    \cos \frac{\theta_{nn}}{2} = \frac{r_{c-nn}}{\sqrt{r^2_{c-nn} + \frac{r^2_{nn}}{4}} }.
\end{align}


\section{Contribution of Nuclear Breakup}
For evaluating the $B(E1)$ value, we need to extract only the Coulomb dissociation component from the experimental data. In this research, we used $\Gamma$ factor method to remove the contribution of nuclear breakup component from lead target. For extracting the Coulomb dissociation cross section $\sigma(CD)$, we subtract the reaction cross section with the carbon target scaled by $\Gamma$ factor from the one with the lead target. Using this method, we can write the Coulomb dissociation cross section as follows.
\begin{align}
    \sigma_{CD} = \sigma(\text{Pb}) - \Gamma \sigma(\text{C}),
\end{align}
$\sigma(\text{C})$ and $\sigma(\text{Pb})$ are the reaction cross section with carbon and lead target respectively. $\Gamma$ is the ratio of the reaction cross section with lead target to the one with carbon target. The $\Gamma$ factor can be obtained from the geometry between the projectile and target nucleus as,
\begin{align}
    \Gamma_{\text{min}} &= \frac{R_{\text{Pb}} + R_{{}^{17}\text{B}}}{R_{\text{C}} + R_{{}^{17}\text{B}}} = \frac{A_{\text{Pb}}^{1/3} + A_{^{17}\text{B}}^{1/3}}{A_{\text{C}}^{1/3} + A_{^{17}\text{B}}} = 1.75\\
    \Gamma_{\text{max}} &= \frac{R_{\text{Pb}}}{R_{\text{C}}} = \frac{A_{\text{Pb}}^{1/3}}{A_{\text{C}}^{C}} = 2.59
\end{align}
In this experiment, we used $\Gamma = 2.385$ value from calculation including the consideration of incident energy of $^{17}$B at the middle of target (270 MeV/u)\cite{Ogata}.

\section{Invariant Mass Method}
To reconstruct the excited state of ${}^{17}$B at target, invariant mass method is used. Since ${}^{17}$B has no bound excited state and its two neutron separation energy $S_{2n}$ is very small, the dissociation process occurs very quickly. In this case, the invariant mass method is a useful tool to reconstruct the intermediate excited state of the system by measuring the momentum and energy of all of the fragments. The invariant mass of the excited state $M^{*}$ is defined as
\begin{align}
    M^* &= \sqrt{\bigg(\sum_{i} E_i\bigg)^2 - \bigg(\sum_{i}\vec{P}_i \bigg)^2} 
\end{align}
where $E_i$ and $\vec{P}_i$ are the energy and momentum of the fragment $i$ respectively. In this experiment, the excited state of ${}^{17}$B is reconstructed by measuring the momentum and energy of ${}^{15}$B and two neutrons. The relative energy $E_{rel}$ between ${}^{15}$B and two neutron can be written with the invariant mass as
\begin{align}
    E_{rel} &= M({}^{17}\text{B}^*) - (m_{{}^{15}\text{B}} + m_n + m_n)
\end{align}
where $m_{{}^{15}\text{B}}$, $m_n$ and $m_n$ are the mass of ${}^{15}$B and two neutrons respectively. The relative energy $E_{rel}$ is related to the excitation energy $E_x$ of ${}^{17}$B and neutron separation energy $S_{2n}$ as
\begin{align}
    E_{rel} &= E_x - S_{2n}
\end{align}
Figure \ref{fig:Invariant_Mass} shows the schematic representation of the invariant mass method. 

\begin{figure}[t]
    \centering
    \setlength{\unitlength}{1mm}
    \begin{picture}(60,40)
        \put(6,15){$E_x$}
        \put(16,1){${}^{17}$B}
        \put(18,31){$M^*$}
        \put(49,11){${}^{15}$B + n + n}
        \put(29,18){$E_{rel}$}
        \put(33,4){$S_{2n}$}
        %\put(10,0){\dashbox{dash-len}}}
        \thicklines
        \put(10,30){\line(1,0){20}}
        \put(10,0){\line(1,0){20}}
        \put(50,10){\line(1,0){20}}
        \put(20,5){\vector(0,1){24}}
        \put(31,29){\vector(1,-1){18}}
        \thinlines
        \multiput(26,10)(1.2,0){20}{\line(1,0){0.8}}
        \multiput(28,0)(1.2,0){10}{\line(1,0){0.8}}
        \put(28,20){\vector(0,1){10}}
        \put(28,20){\vector(0,-1){10}}
        \put(12,20){\vector(0,1){10}}
        \put(12,20){\vector(0,-1){20}}
        \put(32,5){\vector(0,1){5}}
        \put(32,5){\vector(0,-1){5}}
        
    \end{picture}
    \caption{Schematic representation of the invariant mass method} 
    \label{fig:Invariant_Mass}
\end{figure}
\chapter{Experiment}
This chapter describes the experimental setup in this study. The experiment is performed at Rare Isotope Beam Factory (RIBF) at RIKEN Nishina Center \cite{RIKEN}. A primary ${}^{48}$Ca beam is produced by the RIKEN accelerator complex and delivered to the BigRIPS separator \cite{Kubo03}\cite{Kubo07}\cite{Kubo12}. The BigRIPS separator produced the ${}^{17}$B secondary beam which bombarded on the lead and carbon targets in front of the SAMURAI (Superconducting Analyzer for MUlti-particles from Radioisotope beams) spectrometer\cite{SAMURAI}. After the reaction at the target, the fragments ${}^{15}$B and two neutrons are detected by detectors at SAMURAI area. 



The primary beam was accelerated by the SRC, RIBF. By accelerating ${}^{48}$Ca to 345 MeV/u and bombarding it on a thick Be target (30mm), a secondary beam was produced. The characteristics of the primary beam is in Table \ref{tab:Primary_Beam}.

    \begin{table}[h]
        \centering 
            \begin{tabular}[]{c|c|c|c}
                \hline
                Primary Beam & Beam Energy & Beam Intensity & Primary Target  \\
                \hline 
                ${}^{48}$Ca & 345 MeV/u & 210 pnA & Be (30mm)\\
                \hline    
            \end{tabular}
        \caption{Information of Primary Beam}
        \label{tab:Primary_Beam}
    \end{table}
    The collision of the initial beam and the target created the secondary beam, including not only purpose isotope of this experiment, ${}^{19}$B, but also other neighboring isotopes. The main isotope used in this research is ${}^{17}$B, which needed to be separated and identified through the BigRIPS separator. 
    \begin{table}[ht]
        \centering
        \begin{tabular}[ht]{c|c|c}
            \hline
            isotope & Target & Average Energy at the middle of target \\
            \hline
            & C (1.789 g/cm$^2$)  & 270 MeV/u\\
            ${}^{17}$B & Empty  & 275 MeV/u\\
            & Pb (3.255 g/cm$^2$) & 270 MeV/u\\
            \hline    
        \end{tabular}
        \caption{Information of Secondary Beam}
    \end{table}
    The identified ${}^{17}$B was then transferred to SAMURAI, where it underwent Coulomb dissociation, the main objective of this research. Three different targets were used: C, Pb, and Empty. The details for ${}^{17}$B with each target were as follows.

\section{BigRIPS separator}

    \begin{figure}
        \centering
        \includegraphics[width=12cm]{chapter3/BigRIPS_roof.jpg}
        \caption{A top view of the BigRIPS separator \cite{Dayonewiki}}
        \label{fg:BigRIPS}
    \end{figure} 

After the primary ${}^{48}$Ca beam is accelerated at the last superconducting cyclotron SRC, the beam bombarded on a 30mm thick Be target and the secondary beam $^{17}$B is produced through the in-flight fragmentation method. Since the secondary beam includes many rare isotopes, the RI beam is separated by the BigRIPS separator. Table 3.1 shows the BigRIPS separator setup for Dayone experiment. In the first stage of BigRIPS separator, the RI beam separated by dipole magnet with slit and wedge-shaped degrader located at F1 focal plane. In the dipole magnet, the rigidity of a particle can be written as following eq (\ref{eq:rigidity}).
    \begin{align}
        B\rho = v \cdot \frac{A}{Z}  \label{eq:rigidity}
    \end{align}
The velocity of the secondary beam is almost same regardless of the nuclei so that it can be possible to choose specific $A/Z$ by adjusting the slit to filter only some $B\rho$ value. After that, the wedge-shaped degrader makes the beam energy depends on the $Z$ so that even same $A/Z$ nuclei can be separated. In the second stage, the RI beam is identified using TOF-B$\rho$-$\Delta E$ method. Each detector information will be described as following.

    \begin{table}[h]
        \centering
        \begin{tabular}{c c c c}
            \hline
            Location & Dipole [Tm] & Slit[mm] & Degrader / Detector \\
            \hline
            F0  &  &  & Be target (30mm) \\
            F1  &  & $\pm$120 & Al wedge degrader (15mm)\\
            F1-F2 &   8.985  &  & \\
            F2  &  & L: 10 R: 7 & \\
            F3  &  &  & Plastic Scintillator (3mm)\\
            F3-F4 & 8.954 &  & \\
            F4-F5 & 8.96 &  & \\
            F5  &   & $\pm$120 & BPC \\
            F5-F6 & 8.775  &  & \\
            F6-F7 & 8.785  &  & \\
            F7  &  & $\pm$120 & Plastic Scintillator (3mm)\\
            F8 &  & $\pm$170 & \\
            F12 &  & $\pm$170 & \\
            %F13 &  & & Plastic Scintillator (0.5mm$\times$2)\\
            \hline
        \end{tabular}
        \caption{BigRIPS separator setup for Dayone experiment \cite{Dayonewiki}}
    \end{table}

\subsection{Plastic Scintillator}
At focal plane F3, F7, F13, plastic scintillators are located for measuring the time of flight (TOF) of secondary beam. In F3 and F7, plastic scintillator with 3mm thickness is located and at F13 there are two scintillators, SBT1 and SBT2, with 0.5mm thickness. The flight length between F7 and F13 (Average point of SBTs) is 3662.00mm.

\begin{table}[h]
    \centering
    \begin{tabular}{c|cc}
        \hline
        Location & Thickness & Distance from target upstream \\
        \hline
        F3 & 3mm & 86053.56 mm\\
        F7 & 3mm & 39483.58 mm\\
        F13-1 &0.5mm & 2904.08 mm\\
        F13-2 &0.5mm & 2824.08 mm\\
        \hline
    \end{tabular}
    \caption{Information of Plastic Scintillators at F3, F7, F13}
\end{table}


\subsection{BPC (Beam Proportional Chamber)}
BPC is Multi Wire Proportional Chamber (MWPC) located at F5 focal plane which is used for measuring the position of beam. The purpose of BPC is tagging magnetic rigidity and momentum of secondary beam.
\begin{table}[h]
    \centering
    \begin{tabular}{l|c}
        \hline
        Effective Area & (H)240mm $\times$ (V)150mm \\
        Configuration & $XX$ (2 Planes) \\
        number of wire & 64 $\times$ 2 = 128 \\
        Wire Pitch & 4mm \\
        Gas & $i$--${\text{C}}_{4} {\text{H}}_{10}$ at 50 torr\\
        \hline
    \end{tabular}
    \caption{Parameter of BPC (Beam Proportional Chamber) \cite{SAMURAI}}
\end{table}


\begin{figure}[h]
    \centering
    \begin{subfigure}[h]{\textwidth}
        \centering
        \includegraphics[width=12cm]{chapter3/bpc_a1.jpg}
    \end{subfigure}
    \begin{subfigure}[h]{\textwidth}
        \hspace{2.4cm}
        \includegraphics[width=12.5cm]{chapter3/bpc_a23.jpg}
    \end{subfigure}
        \caption{Schematic View of BPC (Beam Proportional Chamber) \cite{SAMURAI}}
\end{figure}

\clearpage

\section{SAMURAI}

\begin{figure}[hbt!]
    \centering
    \includegraphics[width=12cm]{chapter3/SAMURAI1.png}
    \caption{A top view of the beam line from BigRIPS to SAMURAI spectrometer}
\end{figure}

The SAMURAI spectrometer is designed for kinematically complete experiment such as invariant mass spectroscopy. \cite{SAMURAIConcept} Charged fragment bent by SAMURAI superconducting magnet and detected by two drift chambers (FDC1, FDC2) and one plastic scintillator (HODF). Two drift chamber for fragment are located at before and after SAMURAI magnet, for rigidity analysis. And plastic scintillator HODF is placed after FDC2 to measure the TOF and energy loss of fragment. Finally the neutron detector array NEBULA is located at the end of extended beam line for neutron detection. Using SAMURAI system, the invariant mass of the system can be reconstructed by measuring all of the fragments and neutrons.

\begin{figure}[t]
    \centering
    \includegraphics[width=12cm]{chapter3/SAMURAI.png}
    \caption{A top view of the SAMURAI spectrometer}
\end{figure}


\subsection{ICB (Ion Chamber for Beam)}
The ICB is multi-layer ionization chamber for measuring the energy loss ($\Delta E$) of secondary beam. Using P10 gas at 1 atm, the energy loss of secondary beam can be measured. 
\begin{table}[h]
    \centering
    \begin{tabular}{l|c}
        \hline
        Effective Area & (H)140mm $\times$ (V)140mm $\times$ (D)420mm\\
        Configuration & 10 anodes and 11 cathodes \\
        Anode-cathode gap & 21mm \\
        Gas & P10 at 1 atm\\
        Distance from target upstream & 476.87 mm \\
        \hline
    \end{tabular}
    \caption{Parameter of ICB (Ion Chamber for Beam) \cite{SAMURAI}}
\end{table}

\begin{figure}[t]
    \centering
    \includegraphics[width=11cm]{chapter3/icb_a}
    \caption{Schematic View of ICB (Ion Chamber for Beam) \cite{SAMURAI}}
\end{figure}

\subsection{BDC1, BDC2 (Beam Drift Chamber)}
Before target, there are two Beam drift chamber for reconstructing the trajectory of secondary beam. Using the trajectory information, the position of secondary beam at target can be calculated. In this experiment, each BDC box is filled with $i$--${C}_{4} {H}_{10}$ gas at 100 torr. 

\begin{table}[h]
    \centering
    \begin{tabular}[h]{l|c}
        \hline
        Effective Area & (H)80mm $\times$ (V)80mm\\
        Configuration & $XX'YY'XX'YY'$ (8 planes)\\
        Number of Wire & 16 $\times$ 8 = 128 \\
        Wire Pitch & 5mm \\
        Gas & $i$--${\text{C}}_{4} {\text{H}}_{10}$ at 100 torr\\
        Distance from target upstream & (BDC1) 2032.12mm (BDC2) 1032.8 mm \\
        \hline
    \end{tabular}
    \caption{Parameter of BDC (Beam Drift Chamber) \cite{SAMURAI}}
\end{table}

\begin{figure}[h]
    \centering
    \begin{subfigure}{\textwidth}
        \centering
        \includegraphics[width=12.5cm]{chapter3/bdc_a1.jpg}    
    \end{subfigure}
    \begin{subfigure}{\textwidth}
        \hspace{1.5cm}
        \includegraphics[width=12cm]{chapter3/bdc_a3.jpg}
    \end{subfigure}
    \caption{Schematic View of BDC (Beam Drift Chamber) \cite{SAMURAI}}
\end{figure}

\clearpage

\subsection{SBV (Secondary Beam Veto)}

\subsection{Secondary Target}

\begin{table}[h]
    \centering
    \begin{tabular}{c|c}
        \hline
        Material & Thickness \\
        Pb & 3.255 g/cm${}^{2}$\\
         C & 1.789 g/cm${}^{2}$ \\
        \hline
    \end{tabular}
    \caption{Information of Secondary Reaction Target \cite{Dayonewiki}}
\end{table}

\subsection{DALI}

\subsection{SAMURAI Magnet}
Charged particle decayed from ${}^{17}$B is detected by the SAMURAI detector system. After reaction at secondary target, the secondary beam ${}^{17}$B is dissociated to charged particle ${}^{15}$B and two neutrons. The charged particle ${}^{15}$B bent by SAMURAI superconducting dipole magnet. Its trajectory is recorded by FDC1 and FDC2 before and after SAMURAI. And the energy loss and time of flight is measured by HODF plastic scintillator after FDC2.



\begin{table}[h]
    \centering 
    \begin{tabular}{l|c}
    \hline
    Type & Superconducting dipole magnet \\
    Magnet Pole & $\phi$ 2m (0.88 m gap) \\
    Maximum field & 3.1 T \\
    Maximum current & 563 A \\
    Acceptance & $\theta_H \leq \pm 10{}^{\circ}$, $\theta_V \leq \pm 5{}^{\circ}$\\
    \hline
    \end{tabular}
    \caption{Parameter of SAMURAI Magnet \cite{SAMURAI}}
\end{table}

\subsection{FDC1, FDC2 (Forward Drift Chamber)}
After target, there are two Forward Drift Chamber for reconstructing the trajectory of charged fragment. Using the trajectory information, the rigidity of charged fragment at target can be calculated. In this experiment, FDC1 is filled with $i$--${C}_{4} {H}_{10}$ gas at 50 torr and FDC2 is filled with He + 50\% ${C}_{2} {H}_{6}$ gas at 1 atm. 

\begin{table}[h]
    \centering
    \begin{tabular}{l|c}
        \hline
        Effective Area & (H)400mm x (V)300mm x (D)180mm\\
        Configuration & $XX'UU'VV'XX'UU'VV'XX'$ (14 planes)\\
        Number of Wire & 32 $\times$ 14 = 448 \\
        Wire Pitch & 10mm \\
        Gas & $i$--${C}_{4} {H}_{10}$ at 50 torr\\
        Distance from target upstream & 1151.38 mm  \\
        \hline
    \end{tabular}
    \caption{Parameter of FDC1 (Forward Drift Chamber 1) \cite{SAMURAI}}
\end{table}

\begin{table}[h]
    \centering
    \begin{tabular}{l|c}
        \hline
        Effective Area & (H)2296mm x (V)836mm x (D)860mm\\
        Configuration & $XX'UU'VV'XX'UU'VV'XX'$ (14 planes)\\
        Number of Wire & 112 $\times$ 14 = 1568 \\
        Wire Pitch & 20mm \\
        Gas & He + 50\% ${C}_{2} {H}_{6}$ at 1 atm\\
        \hline
    \end{tabular}
    \caption{Parameter of FDC2 (Forward Drift Chamber 2) \cite{SAMURAI}}
\end{table}

\subsection{HODF (HODoscope for Fragment)}
The plastic scintillator HODscope is located behind FDC2 for measuring the TOF and energy loss of charged fragment. For the charged fragment identification, TOF-B$\rho$-$\Delta E$ method is used as same as beam particle identification at BigRIPS. For 
\begin{table}[h]
    \centering
    \begin{tabular}{l|c}
        \hline
        Effective Area & (H)1600mm x (V)1200mm x (D)10mm\\
        Number of Scintillator & 16 \\
        Width of Scintillator & 10mm \\
        \hline
    \end{tabular}
    \caption{Parameter of HODF (HODscope for Fragment) \cite{SAMURAI}}
\end{table}

\subsection{NEBULA}
For measuring momentum vector of neutron, 

\subsection{Geometry Information of SAMURAI Setup}

\begin{figure}
    \centering
    \includegraphics[width=\textwidth]{chapter3/minakata-geometry-drawing_20131121.pdf}
    \caption{Geometry information of SAMURAI setup}
\end{figure}

\section{Run summary}
\begin{center}
    \begin{tabular}[h]{c|ccc}
        \hline
        Run number& Target & Trigger & Note\\
        \hline
        394 - 404 &  C (1.789 g/cm${}^{2}$)  & DSB(1/20) $\cup$  B x N + D(1/1) &\\
        405 - 409 &  Empty  & DSB(1/20) + B x N + D(1/1) &\\
        410 - 427 &  Pb (3.255 g/cm${}^{2}$)  & DSB(1/20) + B x N + D(1/1) &\\
        428, 429, 431 & Pb (3.255 g/cm${}^{2}$)  & DSB(1/1) & F5 slit $\pm$1mm \\
        430 & Pb (3.255 g/cm${}^{2}$)  & DSB(1/1) & F5 slit $\pm$5mm \\
        \hline
    \end{tabular}
\end{center}

\section{Electronics}
\subsection{Data Acquisition System and Trigger condition}
There are four trigger conditions used in this experiment; DSB, B $\cap$ N, B $\cap$ N, B $\cap$ N. They are defined as follows.
\begin{enumerate}
    \item \textbf{DSB} (Down Scale Beam) 
    \item \textbf{B $\cap$ N} (Coincidence between Beam and NEBULA)
    \item \textbf{B $\cap$ D} (Coincidence between Beam and DALI)
    \item \textbf{B $\cap$ H} (Coincidence between Beam and HODF)
\end{enumerate} 

\begin{figure}[h]
    \centering
    \hspace{1cm}
    \includegraphics[width=12cm]{chapter3/Beam_Trigger.png}
    \caption{Logic Diagram for Beam Trigger}
\end{figure}

\begin{figure}
    \centering
    \includegraphics[width=12cm]{chapter3/NEB_Trigger.png}
    \caption{Logic Diagram for Beam Trigger}    
\end{figure}

\begin{figure}
    \centering
    \includegraphics[width=8cm]{chapter3/DALI_Trigger.png}
    \caption{Logic Diagram for Beam Trigger}
\end{figure}

\subsection{Live Time}
The DAQ readout rate is limited by the dead time of the DAQ sub-system. And the dead time depends on the trigger condition and the target.  
\begin{table}[h]
    \centering
    \begin{tabular}{c|c|cc}
        \hline
        Run number & target & DSB & B $\cap$ N, B $\cap$ D \\
        \hline
        394 - 404 & C & 0.843 & 0.815 \\
        405 - 409 & Empty & 0.890 & 0.854 \\
        410 - 427 & Pb & 0.861 & 0.831 \\
        \hline
    \end{tabular}
    \caption{Live time of each reaction trigger}
\end{table}

\chapter{Data Analysis}
In the analysis of experimental data, the primary goal is to extract a differential cross section for Coulomb dissociation of $^{17}$B as a function of relative energy between $^{15}$B and two neutrons. To achieve the goal, I will describe the procedure of identifying the secondary beam of ${}^{17}$B, and selecting events involving the fragment ${}^{15}$B and two neutrons. The flow of the data analysis is as follows.

\begin{center}
    \begin{enumerate}
        \item Select the event containing ${}^{17}$B beam by beam particle identification
        \item Select the event containing ${}^{15}$B fragment by fragment particle identification
        \item Select the event containing two neutrons by cross-talk analysis
        \item Extract the 2$n$ removal cross section of the ${}^{17}\text{B} \to {}^{15}\text{B} + 2n$ reaction
        \item Reconstruct invariant mass at target and obtain the relative energy spectrum 
    \end{enumerate}
\end{center}

\section{Secondary Beam Particle Identification}

The identification of the ${}^{17}$B secondary beam was performed using the TOF-$B\rho$-$\Delta E$ method. Time of Flight (TOF) is obtained from time difference between scintillator at F7 and F13, $B\rho$ is calculated from the beam passing $x$ position at F5, and $\Delta E$ is measured from ionization chamber ICB. $A/Z$ and $Z$ of the secondary beam is driven by the following equation.

\begin{align}
    &\beta_{\text{TOF}_{\text{F7-F13}}} = L(\text{F7-F13}) / ( {\text{TOF}}_{\text{F7-F13}} \times c )\\
    &\beta_{\text{F5}} = f(\text{TOF}_{\text{F7-F13}})\\
    &A/Z = \frac{c \times B\rho_{\text{F5}} \times \gamma_{\text{F5}} }{ m_u \times \beta_{\text{F5}}}
\end{align}
\begin{align}
    Z = \beta_{\text{F7-F13}} \sqrt{\Delta E_{\text{ICB}} \bigg\{ 0.307075 \cdot \Delta x \bigg(\frac{Z_{\text{P10}}}{A_{\text{P10}}}\bigg) \ln \bigg( \frac{2m_{e}c^{2}\beta_{\text{F7-F13}}^{2}\gamma_{\text{F7-F13}}^{2}}{I_{\text{P10}}} - \beta_{\text{F7-F13}}^{2}\bigg) \bigg\}^{-1} }
\end{align}
$L(\text{F7-F13})$ is a distance between F7 and F13 scintillator, $f(\text{TOF}_{\text{F7-F13}})$ is 2nd order polynomial function of TOF for fitting $\beta_{\text{F5}}$. $\Delta E_{\text{ICB}}$ is total energy loss at ICB and $Z_{\text{P10}}$, $A_{\text{P10}}$ and $I_{\text{P10}}$ are effective atomic number, effective mass number and mean excitation energy of P10 gas. $\Delta x$ is a travel distance in ICB. The detail of each steps are described in following.

\subsection{Time of Flight}
The Time of Flight (TOF) is measured using the time difference between two plastic scintillators. In the present analysis, from the timing at F7 plastic scintillator ($t_{\text{F7}}$) and the ones from SBT1,2 at F13 ($t_{\text{SBT1}}$, $t_{\text{SBT2}}$), the TOF$_{\text{F7}-\text{F13}}$ is obtained by following equation.
    \begin{align}
        \text{TOF}_{\text{F7-F13}} = \frac{t_{\text{SBT1}} + t_{\text{SBT2}}}{2} - t_{\text{F7}} + \Delta t_{offset}
    \end{align}
$\Delta t_{offset}$ is the offset used to correct for the difference between the actual measured $\text{TOF}_{\text{F7-F13}}$ and the calculated TOF value with the consideration of energy loss at the materials between F7 and F13. The calculated $\text{TOF}_{\text{F7-F13}}$ value is 192.34 ns, and the corresponding $\Delta t_{offset}$ value is 172.03 ns.

\subsection{Magnetic Rigidity}
Magnetic Rigidity $B\rho$ is derived by Beam Projection Chamber (BPC) located at F5 dispersive focal plane. The $x$ position of a beam passing through F5 is measured by BPC, and the $B\rho$ is calculated using the following equation.
    \begin{align}
        B\rho = (1+\frac{x}{D}) B\rho_{0} 
    \end{align}
with the rigidity of the central trajectory $B\rho_{0}$ is 8.780 Tm, and  momentum dispersion $D$ is 3300 mm/$\%$. 

\subsection{Energy Loss}
The energy loss $\Delta E$ is measured in the Ionize Chamber for Beam (ICB). The correlation between $\Delta E$ and $Z$ can be obtained according to the simplified Bethe-Bloch's energy loss formula as follows.
    \begin{align}
        \frac{\Delta E}{\Delta x} = 2\pi N_{a} r_{e}^{2} m_{e} c^{2} \rho_{\text{P10}} 
        \bigg( \frac{Z_{\text{P10}}}{A_{\text{P10}}} \bigg) \bigg( \frac{Z^{2}}{\beta^{2}} \bigg) 
        \left[ \ln \frac{2m_{e}c^{2}\beta^{2}\gamma^{2}}{I_{\text{P10}}} - \beta^{2}  \right]
    \end{align}
with
    \begin{align}
        2 \pi N_{a} r_{e}^{2} m_{e} c^{2} \rho_{\text{P10}} = 0.307075 \text{ MeV cm}^{2} \text{g}^{-1} 
    \end{align}
    \begin{adjustwidth}{1cm}{}
        $N_{a}$ : Avogadro's number = 6.022 $\times$ 10$^{23}$\\
        $r_{e}$ : classical electron radius = 2.817 $\times$ 10$^{-13}$ cm\\ 
        $m_{e}$ : electron mass = 0.511 MeV/c$^{2}$\\
        $\rho_{\text{P10}}$ : density of the P10 gas = 1.84 $\times$ 10$^{-3}$ g/cm$^{3}$\\
        $Z_{\text{P10}}$ : effective atomic number of the P10 gas\\
        $A_{\text{P10}}$ : effective mass number of the P10 gas\\
        $I_{\text{P10}}$ : mean excitation energy of the P10 gas\\ 
        $\beta$ = $v / c$ of the beam particle\\
        $\gamma$ = $1 / \sqrt{1-\beta^{2}}$
    \end{adjustwidth}
\vspace{3mm}
In this formula, the density effect correction $\delta$ or shell correction $C$ are skipped. Since the P10 gas is compound of 90$\%$ Ar and 10$\%$ CH$_{4}$, the mean excitation energy $I_{\text{P10}}$ is calculated as described in the Appendix A.
The travel distance in ICB, which is calculated as (total length of ICB) $\times$ (probability of each gas in P10) $\times$ (volume density).
\begin{align}
    \Delta x &= 51\times0.9\times0.0016608 + 51\times0.1\times0.000667 
\end{align}

\subsection{Beam Particle Identification}
Figure \ref{fig:Beam_PID} shows the histogram of the particle identification of the secondary beam, showing $Z$ versus $A/Z$. The total numbers of the secondary beam for each target is summarized in Table \ref{tab:Beam_PID}.

\begin{figure}[t]
    \centering
    \includegraphics[width=0.8\textwidth]{chapter4/beampid.png}
    \caption[Secondary beam particle identification]{Beam particle identification of the secondary beam}
    \label{fig:Beam_PID}
\end{figure}

The gate condition for each isotopes are as follows.
\begin{itemize}
    \item DSB trigger 
    \item effective area of target = $x$ $\pm$ 35 mm and $y$ $\pm$ 35 mm
    \item ${}^{17}$B : $Z = 5 \pm 1$ and $A/Z = 17 \pm 1$
    \item ${}^{19}$B : $Z = 5 \pm 1$ and $A/Z = 19 \pm 1$
    \item ${}^{20}$C : $Z = 6 \pm 1$ and $A/Z = 20 \pm 1$
\end{itemize}


\begin{table}[h]
    \centering
    \begin{tabular}{cccc}
        \hline
        Secondary Beam & Pb target & C target & Empty target\\             
        \hline
        ${}^{17}$B & 829586 & 756021 & 331445 \\
        ${}^{19}$B &  160905&  144243&  63033\\
        ${}^{20}$C &  1675080 & 1483113 & 510410 \\
        \hline
    \end{tabular}
    \caption{Statistic of Secondary Beam}
    \label{tab:Beam_PID}
\end{table}

%----------------------------------------------------------------------------------------------------

\section{Beam Profile at Target}

The beam profile at the target can be determined using the two drift chambers, BDC1 and BDC2, located upstream of the target. The incident position and angle at the target are obtained from the beam positions at BDC1 and BDC2.

\subsection{BDC Calibration}
The BDC drift chamber is designed for tracking the incident particle. Beam particle trajectory is obtained by following procedure.

\begin{enumerate}
    \item Obtain a drift time from TDC distribution.
    \item Extract a hit position of each layer from STC (Space to Time Conversion) function.
    \item Fit the trajectory with the linear function by the least-square method.
\end{enumerate}

\subsubsection{TDC (Time to Digital Converter) Distribution}
The timing information of the BDC is obtained by TDC (Time to Digital Converter). In figure \ref{fig:TDC_BDCs}, the TDC distribution of BDC1 and BDC2 are shown. Since we used common stop mode to take a TDC data in this experiment, the drift time is,
\begin{align}
    t_{drift} = t_{max} - t_{\text{TDC}}
\end{align}
where $t_{max}$ is the maximum TDC value. This TDC distribution is obtained from run 431 with $\pm$ 5mm slit at F5. 

\subsubsection{STC (Space to Time Conversion) Function}
The distance between hit position to an anode wire is given by space time conversion (STC) from the drift time. Assuming the uniform position distribution in each drift length cell as
\begin{align}
    \frac{dN}{dx} = const.  
\end{align}
The STC function is derived by integrating the TDC distribution as,
\begin{align}
    &\frac{dN}{dt} \cdot \frac{dt}{dx} = const.\\
    &dx = C \cdot \frac{dN}{dt} \cdot dt,\\
    &x(t) = C \cdot \int_{t_{0}}^{t} \frac{dN}{dt} dt \label{eq:stc}
\end{align}
where, C is normalization factor, $t_0$ is the minimum drift time and $t$ is the drift time from TDC distribution. 
The integration range of TDC for each BDC is shown in table \ref{tab:TDC_BDCs}. The upper and lower limit is determined by the TDC distribution. (Figure \ref{fig:TDC_BDCs}) 
\begin{table}[h]
    \centering
    \begin{tabular}{c|cc}
        \hline
        &Lower limit [ch]& Upper limit [ch]\\
        \hline
        BDC1&640&760\\
        BDC2&630&750\\        
        \hline
    \end{tabular}
    \caption[TDC integration ranges of BDCs]{TDC integration ranges of BDC1 and BDC2}
    \label{tab:TDC_BDCs}
\end{table}

\begin{figure}
    \centering
    \includegraphics[width=14cm]{chapter4/BDCs_TDC.png}
    \caption[TDC Distribution of BDCs]{TDC Distribution of BDC1 (left) and BDC2 (right)}
    \label{fig:TDC_BDCs}
\end{figure}

\subsubsection{Linear fitting of trajectory}
After getting the STC function, the hit position of each layer is calculated from drift time as eq. (\ref{eq:stc}). Now we can get the trajectory of beam particle by fitting the hit position with linear function. When fitting the trajectory, the least-square method is used. The least-square method is the method of finding the best fit of a set of hit positions by minimizing the $\chi^2$ value. The $\chi^2$ value is defined as,
\begin{align}
    \chi^2 = \sum_{i=1}^{N} \frac{(x_{i} - f(x_{fit}))^2}{\sigma_{i}^2}
\end{align}
where, $x_{i}$ is the hit position of each layer, $f(x_{fit})$ is the position from fitting function, $\sigma_{i}$ is the position resolution of each layer.

\subsubsection{The resolution of BDCs}
After tracking the trajectory, the resolution of BDCs can be evaluated by the tracking residue distribution. The tracking residue $x_{residue}$ is defined as,

\begin{align}
    x_{residue} = x_{trac} - x_{drift}
\end{align}

where $x_{trac}$ is the calculated position from the trajectory and $x_{drift}$ is the hit position of each layer. The tracking residue distribution of BDC1 and BDC2 are shown in Figure \ref{fig:residue_bdcs}. The width of residue distribution for $x$ and $y$ direction are $\Delta x$ = 0.2605 mm, $\Delta y$ = 0.2632 mm respectively. From this, the position and angular resolution are calculated as described in the Appendix B. The result is described in table \ref{tab:resolution_bdcs}.
 \begin{figure}
    \centering
    \includegraphics[width=13cm]{chapter4/BDC_residu.png}
    \caption[Tracking Residue Distribution of BDCs]{Tracking Residue Distribution of BDC1 (left) and BDC2 (right)}
    \label{fig:residue_bdcs}
\end{figure}

\begin{table}[b]
    \centering
    \begin{tabular}{c|cc}
    \hline
     & $\sigma_x$ [mm] & $\sigma_y$ [mm]\\
    \hline
    BDC1 & 0.1340 & 0.1627 \\
    BDC2 & 0.1422 & 0.1743 \\
    \hline 
    \hline
    & $\sigma_a$ [rad] & $\sigma_b$ [rad]\\
    \hline
    BDC1 & 0.01316 & 0.01330 \\
    BDC2 & 0.01396 & 0.01424 \\
    \hline
    \end{tabular}
    \caption{Position and Angular Resolution of BDCs}
    \label{tab:resolution_bdcs}
\end{table}

\subsection{Beam Profile at Target}
The beam profile at the target is obtained by using the position information from BDCs. The position information ($x_{\text{tgt}}$, $y_{\text{tgt}}$) and angle information ($a_{\text{tgt}}$, $b_{\text{tgt}}$) at the target are obtained as,

\begin{align}
    x_{\text{tgt}} &= x_{\text{BDC1}} + \frac{(x_{\text{BDC2}} - x_{\text{BDC1}})}{L(\text{BDC1} - \text{BDC2})} \times L(\text{BDC1} - \text{tgt})\\
    y_{\text{tgt}} &= y_{\text{BDC1}} + \frac{(y_{\text{BDC2}} - y_{\text{BDC1}})}{L(\text{BDC1} - \text{BDC2})} \times L(\text{BDC1} - \text{tgt})\\
    a_{\text{tgt}} &= tan^{-1} \bigg( \frac{(x_{\text{BDC2}} - x_{\text{BDC1}})}{L(\text{BDC1} - \text{BDC2})} \bigg)\\
    b_{\text{tgt}} &= tan^{-1} \bigg( \frac{(y_{\text{BDC2}} - y_{\text{BDC1}})}{L(\text{BDC1} - \text{BDC2})} \bigg),
\end{align}
where the distance between BDCs is $L(\text{BDC1} - \text{BDC2}) = 999.3 $mm, and the distance between BDC1 and the target is $L(\text{BDC1} - \text{tgt}) = 2017.1$mm. The beam profile at the target is shown in Figure \ref{fig:beam_profile_tgt}.
\begin{figure}[h]
    \centering
    \includegraphics[width=10cm]{chapter4/target.png}
    \caption{${}^{17}$B Beam profile at target}
    \label{fig:beam_profile_tgt}
\end{figure}


\begin{align}
    \Delta \theta_x^{\text{tgt}} = \Delta \bigg( \frac{x_{\text{BDC2}} - x_{\text{BDC1}}}{L(\text{BDC2} - \text{BDC1})} \bigg) 
    = \frac{\sqrt{(\Delta x_{\text{BDC2}})^2 + (\Delta x_{\text{BDC1}})^2}}{1000 \text{mm}} \\ \notag
    \Delta \theta^{\text{tgt}}_y = \Delta \Big( \frac{y_{\text{BDC2}} - y_{\text{BDC1}}}{L(\text{BDC2} - \text{BDC1})} \Big) 
    = \frac{\sqrt{(\Delta y_{\text{BDC2}})^2 + (\Delta y_{\text{BDC1}})^2}}{1000 \text{mm}}
\end{align}
The position resolution at the target is
\begin{align}
    \Delta x_{\text{tgt}} = \sqrt{(\Delta x_{\text{BDC2}})^2 + (\Delta \theta_x^{\text{tgt}} \cdot L(\text{tgt} - \text{BDC2}))^2} \\ \notag
    \Delta y_{\text{tgt}} = \sqrt{(\Delta y_{\text{BDC2}})^2 + (\Delta \theta_y^{\text{tgt}} \cdot L(\text{tgt} - \text{BDC2}))^2}
\end{align}

\clearpage

\section{Charged Particle Identification}

Charged particles are identified using the TOF-B$\rho$-$\Delta$E method as same as the beam particle identification. Time of Flight is obtained from time difference between target and HODF, $B\rho$ is calculated from FDC1 and FDC2 positions and angles, and $Z$ is calculated by HOD $Q$ information. $A/Z$ and $Z$ is obtained by the following equation.

\begin{align}
    &\beta_{\text{frag}} = L(\text{tgt - HODF}) / ( {\text{TOF}}_{\text{tgt - HODF}} \times c )\\
    &A/Z = \frac{c \times B\rho \times \gamma_{\text{frag}}} { m_u \times \beta_{\text{frag}}}\\
    &Z = p_0 + p_1 (Q_{\text{HOD}} - p_2 \frac{1}{\beta^2})
\end{align}

$L$(tgt - HODF) is flight length from target to HODF. This is also calculated by a function of the positions and angles at FDC1 and FDC2 as the one for $B\rho$. The coefficient of $Z$ calculation $p_0, p_1, p_2$ are obtained by linear fitting of HOD $Q$ distribution. The detail of each steps will be described in following.

\subsection{FDC Calibration}
The tracking procedure of FDCs is same as BDC. The integration range of TDC for each FDC is shown in table \ref{tab:TDC_FDCs} And the TDC distribution for determining the upper and lower limit are shown in Figure \ref{fig:TDC_FDCs}. 
\begin{table}[h]
    \centering
    \begin{tabular}{c|cc}
        \hline
        &Lower limit [ch]&Upper limit [ch]\\
        \hline
        FDC1&1300&1600\\
        FDC2&500&1470\\        
        \hline
    \end{tabular}
    \caption[TDC integration range of FDCs]{TDC integration ranges of FDC1 and FDC2}
    \label{tab:TDC_FDCs}
\end{table}

\begin{figure}
    \centering
    \includegraphics[width=14cm]{chapter4/FDCs_TDC.png}
    \caption[TDC Distribution of FDCs]{TDC Distribution of FDC1 (left) and FDC2 (right)}
    \label{fig:TDC_FDCs}
\end{figure}

\subsubsection{The resolution of FDCs}
The residue distribution of each $X$, $U$, $V$ plane is shown in Figure \ref{fig:residue_fdcs}. The width of residue distribution for $x$ and $y$ direction are $\Delta x$ = 0.2605 mm, $\Delta y$ = 0.2632 mm respectively. From this, the position and angular resolution are calculated as described in the Appendix B. The result is described in table \ref{tab:resolution_fdcs}.
 \begin{figure}
    \centering
    %\includegraphics[width=13cm]{chapter4/FDC_residu.png}
    \caption[Tracking Residue Distribution of FDCs]{Tracking Residue Distribution of FDC1 (left) and FDC2 (right)}
    \label{fig:residue_fdcs}
\end{figure}

\subsection{Magnetic Rigidity}
The Brho of the charged particle is calculated from the positions and angles obtained from FDC1 and FDC2. 
\begin{align}
    B\rho &= f(x_{\text{FDC1}}, y_{\text{FDC1}}, a_{\text{FDC1}}, b_{\text{FDC1}}, x_{\text{FDC2}}, a_{\text{FDC2}})\\
    &= \sum_{i} c_{1,i} a_i + \sum_{i}\sum_{j} c_{2,ij} a_i a_j + \sum_{i}\sum_{j}\sum_{k} c_{3,ijk} a_i a_j a_k + \cdots \\
    &= c_{1,0} x_{\text{FDC1}} + c_{1,1} y_{\text{FDC1}} + \cdots + c_{2,00} x_{\text{FDC1}}^2 + c_{2,01} x_{\text{FDC1}} y_{\text{FDC1}} + \cdots + c_{3,000} x_{\text{FDC1}}^3 + \cdots
\end{align}
The function of B$\rho$ is extracted by using TMultiDimFit class in ROOT using a trajectory obtained from Geant4 simulation.
\begin{align}
    &a_{\text{FDC1}} = tan^{-1} \bigg( \frac{x_{\text{FDC1}} - x_{\text{tgt}}}{L(\text{FDC1} - \text{target})} \bigg)\\
    &b_{\text{FDC1}} = tan^{-1} \bigg( \frac{y_{\text{FDC1}} - y_{\text{tgt}}}{L(\text{FDC1 - target})} \bigg)
\end{align}

\subsection{Time of Flight and Energy Loss}
\subsubsection{Time of Flight}
Time of flight of charged particle is determined by the time difference between the HODF and the target. The time at target is determined by addition of the time at F13 and the time of flight from F13 to the target. The time of flight between F13 and the target is calculated with ${}1^{17}$B beam with Energy loss calculation considered the material between F13 and the target. The time of flight between HOD and target is defined by the following equation.

\begin{align}
    t_{\text{tgt}} = t_{F13} + tof_{\text{F13-tgt}}
    tof_{\text{HODF-tgt}} = t_{\text{HODF}} - t_{\text{tgt}} + \Delta t_{offset}
\end{align}
where $t_{\text{HOD}}$ is timing information of HODF and $\Delta t_{offset}$ is obtained by Geant4 simulation as same as B$\rho$ analysis. The $\Delta t_{offset}$ is 112.5256 ns.

\subsubsection{Energy Loss}
In case of fragment, the energy loss is obtained from the light output information of HODF scintillator. From the Bethe-Bloch formula, we assumed that 
\begin{align}
    Z_{frag} \propto \beta_{frag} \sqrt{\Delta E} = \beta_{frag} \sqrt{Q_{\text{HOD}}}
\end{align}
Because of the proportional relation between $Z_{frag}$ and $\sqrt{Q_{\text{HOD}}}$, it is difficult to identify the fragment particle by only energy loss information. So first we gate the fragment particle by ${}^{17}$B beam, and assumed that most of fragment came from ${}^{17}$B should be ${}^{17}$B itself. Then we can get coefficient for calculating the Z.
\begin{align}
\end{align}

\subsection{Fragment Particle Identification}
The fragment particle identification is shown in figure \ref{fig:fragpid_all} and \ref{fig:fragpid_b17}. Figure \ref{fig:fragpid_all} is the fragment particle identification of all events. Figure \ref{fig:fragpid_b17} is the fragment particle identification only from ${}^{17}$B beam events.
\begin{figure}
    \centering
    \includegraphics[width=14cm]{chapter4/fragpid_all.png}
    \caption[Fragment Particle Identification from All Secondary Beam]{Fragment Particle Identification of All Events at Pb target (left) and C target (right)}
    \label{fig:fragpid_all}
\end{figure}

\begin{figure}
    \centering
    \includegraphics[width=14cm]{chapter4/fragpid_b17.png}
    \caption[Fragment Particle Identification from ${}^{17}$B Beam]{Fragment Particle Identification of ${}^{17}$B Events at Pb target (left) and C target (right)}
    \label{fig:fragpid_b17}
\end{figure}

\clearpage

\section{Analysis of Neutrons}
In this experiment, neutrons emitted from the secondary beam ${}^{17}$B are detected by the NEBULA neutron detector array. A neutron is detected indirectly by recoiled proton which is mainly produced by H($n$,$n$) and ${}^{12}$C($n$,$np$) reaction in the plastic scintillator. Because of the indirect detection of neutrons, the selection of neutron events is more complicated than that of charged particles. The selection of neutron events is performed in three steps. First, reject the events which are considered to be a charged particle or gamma event. Second, in two neutron selection case, remove the event which is considered to be a cross-talk. After all, select the fastest and second fastest event as a real neutron event. In following section, the detail of each step is described.

\subsection {Selection of Real Neutron Events}
%In the selection procedure of real neutron event, the rejection of charged particle and gamma ray is almost same with one neutron selection. But in case of selecting events of two neutrons, the most important process is the elimination of cross-talk. Cross-talk refers to the phenomenon where a single neutron generates multiple signals. Cross-talk is the most significant source of noise when selecting two-neutron events. The selection of two neutron events is performed in five steps. 
\begin{enumerate}
    \item All events detected by the first VETO are considered to be charged particles and are rejected.
    \item Among the events detected by NEBULA, events with a light output Q of less than 6 MeVee are considered to be gamma rays and are rejected. In addition, events exceeding the maximum energy loss of a recoiled proton in one scintillator unit (130 MeV) are also rejected.
    \item Events whose TOF from the target to the first wall is less than 40 ns and whose TOF from the target to the second wall is less than 42 ns are considered to be non-neutron events with $\beta < 0.9$ and are rejected.
    \item (For the selection of two neutron events) For events detected at the second VETO, detections in which the two fastest neutron events incident on the second NEUT wall are dr(xy) $<$ 500mm and 1ns $<$ dt $<$ 5ns are considered to be cluster scattering events from second VETO and the second event is rejected.
    \item (For the selection of two neutron events) Cross-talk events are rejected.
\end{enumerate}
I will describe the three step of cross-talk rejection.

\subsection{Cross-talk Rejection}
To reject cross-talk, a Geant4 simulation was performed to generate events of ${}^{16}\text{B} \to {}^{15}\text{B}+n$, thereby replicating cases where all two-neutron events are due to cross-talk. The details of the executed Geant4 simulation are shown in Table \ref{tab:cross-talk_sim}.

\begin{table}
    \centering
    \begin{tabular}[h]{c|c}
        \hline 
        Reaction & ${}^{16}\text{B} \to {}^{15}\text{B}+n$ \\
        Beam Energy & 270 MeV/u\\
        Relative Energy & 0 - 10 MeV (Uniformly Generated)\\
        Position Distribution & Reconstructed from ${}^{17}$B Beam profile\\
        Angular Distribution & Reconstructed from ${}^{17}$B Beam profile \\
        \hline
    \end{tabular}
    \caption{The Geant4 simulation condition for cross-talk rejection}
    \label{tab:cross-talk_sim}
\end{table}

\subsubsection{Clustering event}
The clustering event means the two events which are detected in very close distance in very small time interval. It means the second event is likely to be a recoil proton from the first event. The clustering event is rejected by the following condition.

\begin{figure}[h]
    \centering
    \includegraphics[width=8cm]{chapter4/drdt.png}
    \caption[Clustering Event]{Clustering Event}
    \label{fig:clustering}
\end{figure}

\begin{align}
    \bigg( \frac{dr-dr_0}{3\sigma_{dr}} \bigg)^2 + \bigg( \frac{dt-dt_0}{3\sigma_{dt}} \bigg)^2 < 1 
\end{align}
where $dr$ and $dt$ are the position and time difference between two events. And $dr_0$ and $dt_0$, $\sigma_{dr}$ and $\sigma_{dt}$ mean the central values and the standard deviations of the distribution of $dr$ and $dt$ for the clustering events. The $dr_0$ and $dt_0$ are 98.35 mm and 0.65 ns, and $\sigma_{dr}$ and $\sigma_{dt}$ are 71.07 mm and 0.40 ns. 

\subsubsection{same wall event}
For the rejection of cross talk in the case of two event is hit on the same wall, the relative velocity between the events is used. After the selection of two neutron events, each event is tagged by the hit order. The light output is tagged as $Q_1$ and the one of second event is tagged as $Q_2$. The relative velocity between two events is defined as $\beta_{01}/\beta{12}$ where $\beta_{01}$ represents the velocity between first hit and the target and $\beta_{12}$ represents the velocity between first and second hit. Figure 00 shows the distribution of light output of second event $Q_2$ with the function of relative velocity $\beta_{01}/\beta{12}$.
\begin{figure}[h]
    \centering
    \includegraphics[width=0.4\textwidth]{chapter4/same1.png}\hspace{0.5cm}
    \includegraphics[width=0.4\textwidth]{chapter4/same2.png}
    \caption[Cross-talk Rejection for Same Wall Event]{Cross-talk Rejection for Same Wall Event}
    \label{fig:samewall}
\end{figure}

In the figure, the events with $\beta_{01}/\beta_{12} > 1$ are considered to be cross-talk events because the scattered neutron from first hit is considered to be slower than the first hit. Also, the ev
\subsubsection{different wall event}

\begin{figure}
    \centering
    \includegraphics[width=0.4\textwidth]{chapter4/diff1.png}\hspace{0.5cm}
    \includegraphics[width=0.4\textwidth]{chapter4/diff2.png}
    \caption[Cross-talk Rejection for Different Wall Event]{Cross-talk Rejection for Different Wall Event}
    \label{fig:differentwall}
\end{figure}

\subsection{Cross-talk Residual Rate}
Even though we performed the cross talk rejection, there probably be residual of cross talk. For each cross talk step, we evaluated the residual rate of cross talk. Each step is as follows.
\begin{enumerate}
    \item (a) no rejection
    \item (b) clustering rejection
    \item (c) clustering rejection + same wall rejection
    \item (d) clustering rejection + same wall rejection + gamma rejection
\end{enumerate}
The residual rate of cross talk can be calculated by the following formula.
\begin{align}
    R = \frac{N_{M>2}}{N_{M>1}}
\end{align}
The result is the multiplicity M $\>$ event is 225385, and M>2 event of each step and residual rate R are in table 4.7.
\begin{table}[h]
    \centering
    \begin{tabular}[h]{c|c|c|c}
        \hline
        Condition & same wall event (R) & different wall event (R) & all wall event (R)\\
        \hline
        (a) & 89721 (39.8$\%$) & 10123 (4.5$\%$) & 99844 (44.3$\%$) \\
        (b) & 29032 (12.9$\%$) & 16848 (7.5$\%$) & 45880 (20.4$\%$)\\
        (c) & 6274 (2.8)$\%$)   & 2791 (1.2$\%$)& 9065 (4.0$\%$)\\
        (d) & 5352 (2.4$\%$)& 2089 (0.9$\%$)& 7441 (3.3$\%$)\\
        \hline
    \end{tabular}
    \caption{The cross talk residual rate evaluation}
\end{table}

\section{Acceptance and Efficiency Correction}
For evaluating the two-neutron detection efficiency and SAMURAI acceptance, the simulation is performed by Geant4. The information of simulation is as follows.

\begin{center}
    \begin{tabular}[h]{c|c}
        \hline
        Physics Model & Phase Space Decay \\
        Reaction & ${}^{17}\text{B} \to {}^{15}\text{B} + 2n$\\
        Beam Energy & 270 MeV/u\\
        Relative Energy & 1-10 MeV (Uniformly generated)\\
        Scattering Angle & 0-30 mrad (Uniformly generated)\\
        \hline
    \end{tabular}
\end{center}

The result of simulation is shown in figure \ref{fig:acc_same_diff}.
\begin{figure}
    \centering
    \includegraphics[width=14cm]{chapter4/acc_same_diff.png}
    \caption[$2n$ Acceptance for $E_{rel}$ and $\theta_{scat}$]{2n Acceptance for same wall (left) and different wall (right)}
    \label{fig:acc_same_diff}
\end{figure}


\section{Reconstruction of Invariant Mass}

After identification of all fragments from the ${}^{17}$B beam, the invariant mass of the system can be reconstructed. First, the beam vector is reconstructed by $\beta_{\text{Beam}}$ and beam profile at target. $\beta_{\text{Beam}}$ is calculated using Eloss code, considering the energy loss from F5 slit to the middle of target.

\begin{align}
    %\gamma_{\text{Beam}} &= \frac{1}{\sqrt{1-\beta_{\text{Beam}}^{2}}}\\
    P_z ({}^{17}\text{B}) &=  m_{{}^{17}\text{B}} \gamma \frac{1}{\sqrt{1+ \tan^2(\text{tgt}a)+ \tan^2(\text{tgt}b)}}\\
    P_x ({}^{17}\text{B}) &=  P_z ({}^{17}\text{B}) \cdot \tan(\text{tgt}a)\\ 
    P_y ({}^{17}\text{B}) &=  P_z ({}^{17}\text{B}) \cdot \tan(\text{tgt}b)
\end{align}

Second, the momentum of the ${}^{15}$B fragment is calculated from the 
\begin{align}
    \vec{P} ({}^{15}\text{B}) &= B\rho c Z \\
    E ({}^{15}\text{B}) &= \sqrt{m({}^{15}\text{B})^{2} + P({}^{15}\text{B})^{2}}\\
    \vec{P_z} ({}^{15}\text{B}) &= \vec{P} \frac{1}{\sqrt{1 + \tan^2(scata)+ \tan^2(scatb)}}\\
    \vec{P_x} ({}^{15}\text{B}) &= \vec{P_z} ({}^{15}\text{B}) \cdot \tan(scata)\\
    \vec{P_y} ({}^{15}\text{B}) &= \vec{P_z} ({}^{15}\text{B}) \cdot \tan(scatb)
\end{align}
Finally the neutron vector is reconstructed by the position information from NEBULA and the time-of-flight as (4.34). From these three vectors, the invariant mass of the system is calculated.
The momentum $P(n)$ of the neutron is reconstructed from the detection position in NEBULA relative to the target, and the time-of-flight (TOF). The momentum vector of the neutron is described as follows.

\begin{align}
    L &= | \vec{r}_{\text{tgt}} - \vec{r}_{n} | \\
    \beta_{n} &= L / (\text{TOF}_{\text{NEB-tgt}} \times c) \\
    P_{n} &= m_{n} \beta_{n} \gamma_{n} \\
    E_{n} &= m_{n} \gamma_{n} \\
    \vec{P_{n}} &= \frac{\vec{r_{n}} - \vec{r_{tgt}}}{L} P_{n}
\end{align}
where, $\vec{r}_{\text{tgt}}$ is the position of the target $\vec{r}_{\text{tgt}}$, $m_{n}$ is the neutron mass


\section{Relative Energy Spectrum}
\subsection{Relative Energy Spectrum of ${}^{15}\text{B} + n + n$}

\begin{figure}[h]
    \centering
    \includegraphics[width=14cm]{chapter4/Pb_same_diff.png}
    \caption{Relative Energy Spectrum of ${}^{15}\text{B} + 2n$ at Pb target}
\end{figure}
\begin{figure}[h]
    \centering
    \includegraphics[width=14cm]{chapter4/C_same_diff.png}
    \caption{Relative Energy Spectrum of ${}^{15}\text{B} + 2n$ at C target}
\end{figure}

%\begin{figure}
%    \centering
%    \includegraphics[width=14cm]{chapter4/16Bn_C.png}
%    \caption{Relative Energy Spectrum of ${}^{16}\text{B} + n$ at C target}
%\end{figure}

\subsection{Relative Energy Spectrum of ${}^{15}\text{B} + n$}

\chapter{Result and Discussion}
In this chapter, we discuss about the results obtained through data analysis.
\section{Inclusive Cross Section}

\section{Neutron Removal Cross Section}
\begin{table}[h]
\centering
\begin{tabular}{c|c|c|c}
    \hline
    & $\sigma_{incl}$ (mb) & $\sigma_{-2n}$ (mb) & $\sigma_{-4n}$ (mb) \\
    \hline \hline
    $^{17}$B + Pb&  & 713() &    \\ 
    $^{17}$B + C &  & 149() &   \\ 
    $\sigma_{Pb}/\sigma_{C}$ &  &  &    \\ \hline
\end{tabular}
\caption{Reaction Cross Section of Boron Isotopes}
\label{tab:Reaction Cross Section of Boron Isotopes}

\end{table}

\section{Relative Energy Spectrum}

\section{Coulomb Dissociation Cross Section}
Coulomb dissociation cross section can be extracted by using the following equation.
\begin{align}
\sigma_{CD} = \sigma_{incl}(Pb) - \Gamma \sigma_{incl}(C)
\end{align}
We used $\Gamma$ value from calculation . 

\begin{table}[h]
\centering
    \begin{tabular}{c|c|c}
        \hline
        & $\Gamma$ & $\sigma_{coul}$   \\
        \hline
        $^{17}$B& 2.385 &   281 $\pm$ 8 mb \\ 
        \hline
    \end{tabular}
\caption{Coulomb Dissociation Cross Section}
\label{Coulomb Dissociation Cross Section}
\end{table}

\begin{center}
    \includegraphics[width=13cm]{chapter5/coulomb_pb_c.png}    
    \captionof{figure}{Cross Section}
    \includegraphics[width=13cm]{chapter5/dsigmadE.png}    
    \captionof{figure}{Coulomb Dissociation Cross Section}
\end{center}    

About reduced dipole transition probability, we can extract $B(E1)$ strength as follows.
\begin{align}
    \frac{d \sigma_{coul}}{dE_x} = \frac{16 \pi^{3} }{9 \hbar c} N_{\text{E1}}(E_x) \frac{dB(\text{E1})}{dE_x}
\end{align}

\begin{center}
    \includegraphics[width=13cm]{chapter5/dBdEx.png}    
    \captionof{figure}{B(E1) Strength}
\end{center}  
\begin{align}
    B(E1) &= \int_{0}^{+\inf} \frac{dB(E1)}{dE_x} dE_x \\
          &= \int_{0}^{+10} \frac{dB(E1)}{dE_x} dE_x \\ 
          &= 1.22 \pm 0.06  [e^{2}fm^{2}]    
\end{align}  

\section{Dineutron Correlation}

\begin{align}
    B(E1) = \frac{3}{\pi} \bigg( \frac{Ze}{A} \bigg)^2 \langle r^{2}_{c-nn} \rangle
\end{align}
With the obtained $B(E1)$ strength up to 10 MeV, we can extract $\sqrt{ \langle r^{2}_{c-nn} \rangle}$ as follows.
\begin{align}
    \sqrt{ \langle r^{2}_{c-nn} \rangle} = 0.00 \pm 0.00 (stat.) \pm 0.00 (syst.) \text{ fm}
\end{align}
Furthermore, we can extract the distance of $2n$ from three body model.
\begin{align}
    \langle r^{2}_{halo} \rangle = \frac{A_c}{A} \langle r^{2}_{core} \rangle + \frac{2A_c}{A^2} \langle r^{2}_{c-nn} \rangle - \frac{1}{2A} \langle r^{2}_{nn} \rangle
\end{align}
where $A$ and $A_c$ is the mass number of halo nucleus and core. $\langle r^{2}_{h} \rangle$ and $\langle r^{2}_{c} \rangle$ are the mean-square matter radius of halo nuclei and the core, which is 3.00(6) fm and 2.75(6) fm respectively.
\begin{align}
    \sqrt{ \langle r^{2}_{nn} \rangle} = 3.20 \pm 0.00 (stat.) \pm 0.00 (syst.) \text{ fm}
\end{align}
Finally, the mean opening angle of dineutron can be extracted as follows.
\begin{align}
    \langle \theta_{nn} \rangle = 6.57 \pm 0.00 (stat.) \pm 0.00 (syst.) \text{ deg.}
\end{align}
\chapter{Conclusion}
In this thesis, we investigated the Coulomb dissociation of two neutron halo nucleus ${}^{17}$B with carbon and lead target at 270 MeV/u. We extracted the inclusive reaction cross section for two and four neutron removal reaction respectively. According to the 2$n$ removal reaction cross section ratio between lead and carbon target, which is 3.9, the Coulomb dissociation of $^{17}$B at lead target is not to be dominant as the neighboring nucleus $^{19}$B which has a large 2$n$ removal reaction cross section ratio, 7.1\cite{KJCook}. And the Coulomb dissociation cross section is also extracted from relative energy $E_{rel}$ spectrum. The Coulomb dissociation cross section spectrum of $^{17}$ has a broad peak around 2 $\sim$ 3 MeV, which is larger than the one of other halo nuclei such as ${}^{11}$Li\cite{Nakamura06} and $^{19}$B\cite{KJCook}. The peak position of the Coulomb dissociation cross section spectrum is related with the strength of soft $E1$ excitation of halo nuclei, and the broad peak of $^{17}$B is considered to be caused by a weak halo or a neutron skin structure. Also, by integrating the spectrum, the Coulomb dissociation cross section is obtained as 362 $\pm$ 11 mb in a range up to 7 MeV and 438 $\pm$ 13 mb up to 10 MeV. Both are significantly smaller than the one of ${}^{19}$B which is approximately 1 b\cite{KJCook}.\\
By equivalent photon method, we extracted the reduced $E1$ transition probability $B(E1)$ spectrum. The integrated $B(E1)$ value up to 7 MeV was 1.32 $\pm$ 0.06 e$^2$fm$^2$ and the one up to 10 MeV was 2.00 $\pm$ 0.10 e$^2$fm$^2$. The $B(E1)$ value for $^{19}$B was 1.64 $\pm$ 0.06 (\textit{stat}) e$^2$fm$^2$ in a range up to 6 MeV\cite{KJCook} which is larger than the one for $^{17}$B. Also, compared to the shape of $B(E1)$ spectrum for $^{19}$B, the $B(E1)$ spectrum for $^{17}$B has very broad curve and the peak position is around 4 $\sim$ 5 MeV. These features indicate that the Coulomb dissociation is not dominant for $^{17}$B as well as in the case of $^{19}$B. \\
Dineutron correlation is also investigated by the opening angle of valance neutrons. The opening angle of valance neutrons in $^{17}$B was 56.6 $\pm$ 19.4 degree in a range of $B(E1)$ up to 10 MeV, and 87.0 $\pm$ 16.6 degree in a range of $B(E1)$ up to 7 MeV. The average result has consistent with the recent research result, 77.4 degree by A. Corsi\cite{Corsi}. \\
For the future plan, evaluation of systematical error is needed. Also the contribution of excited state of $^{17}$B at the reaction point should be considered by $\gamma$ ray analysis. Also the opening angle of valance neutrons and the dineutron correlation can be calculated with a three-body model can be a theoretical support. 
\appendix

\chapter{Constants}
\section{Mass}
For calculating the mass of the nucleus, the mass excess is used. The mass is defined with mass excess ($\Delta$) as
\begin{align}
    M(A,Z) = A m_u + \Delta(A,Z),
\end{align}
where $m_u$ is the atomic mass unit. In this thesis, the mass excess of each nucleus is used from the ~~
\begin{table}[h]
    \centering
    \begin{tabular}{c|c}
        \hline
        Nucleus & Mass Excess(MeV) \\
        \hline
        ${}^{19}$B & 59.8 \\
        ${}^{17}$B & 43.72 \\
        ${}^{15}$B & 28.957 \\
        ${}^{14}$B & 23.664 \\
        ${}^{13}$B & 16.561 \\
        \hline
    \end{tabular}
\end{table}

\section{}


%\bibliography{reference}

\printbibliography

\backmatter

 

\end{document}
