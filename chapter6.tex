\chapter{Conclusion}
In this thesis, we investigated the Coulomb dissociation of two neutron halo nucleus ${}^{17}$B with carbon and lead target at 270 MeV/u. We extracted the inclusive reaction cross section for two and four neutron removal reaction respectively. According to the 2$n$ removal reaction cross section ratio between lead and carbon target, which is 3.9, the Coulomb dissociation of $^{17}$B at lead target is not to be dominant as the neighboring nucleus $^{19}$B which has a large 2$n$ removal reaction cross section ratio, 7.1\cite{KJCook}. And the Coulomb dissociation cross section is also extracted from relative energy $E_{rel}$ spectrum. The Coulomb dissociation cross section spectrum of $^{17}$ has a broad peak around 2 $\sim$ 3 MeV, which is larger than the one of other halo nuclei such as ${}^{11}$Li\cite{Nakamura06} and $^{19}$B\cite{KJCook}. The peak position of the Coulomb dissociation cross section spectrum is related with the strength of soft $E1$ excitation of halo nuclei, and the broad peak of $^{17}$B is considered to be caused by a weak halo or a neutron skin structure. Also, by integrating the spectrum, the Coulomb dissociation cross section is obtained as 381 $\pm$ 11 mb in a range up to 7 MeV and 458 $\pm$ 14 mb up to 10 MeV. Both are significantly smaller than the one of ${}^{19}$B which is approximately 1 b\cite{KJCook}.\\
By equivalent photon method, we extracted the reduced $E1$ transition probability $B(E1)$ spectrum. The integrated $B(E1)$ value up to 7 MeV was 1.32 $\pm$ 0.06 e$^2$fm$^2$ and the one up to 10 MeV was 2.00 $\pm$ 0.10 e$^2$fm$^2$. The $B(E1)$ value for $^{19}$B was 1.64 $\pm$ 0.06 (\textit{stat}) e$^2$fm$^2$ in a range up to 6 MeV\cite{KJCook} which is larger than the one for $^{17}$B. Also, compared to the shape of $B(E1)$ spectrum for $^{19}$B, the $B(E1)$ spectrum for $^{17}$B has very broad curve and the peak position is around 4 $\sim$ 5 MeV. These features indicate that the Coulomb dissociation is not dominant for $^{17}$B as well as in the case of $^{19}$B. \\
Dineutron correlation is also investigated by the opening angle of valance neutrons. The opening angle of valance neutrons in $^{17}$B was 56.6 $\pm$ 19.4 degree in a range of $B(E1)$ up to 10 MeV, and 87.0 $\pm$ 16.6 degree in a range of $B(E1)$ up to 7 MeV. The average result has consistent with the recent research result, 77.4 degree by A. Corsi\cite{Corsi}. \\
For the future plan, evaluation of systematical error is needed. Also the contribution of excited state of $^{17}$B at the reaction point should be considered by $\gamma$ ray analysis. Also the opening angle of valance neutrons and the dineutron correlation can be calculated with a three-body model can be a theoretical support. 